\documentclass{article}

\usepackage{arxiv}

\usepackage[utf8]{inputenc} % allow utf-8 input
\usepackage[T1]{fontenc}    % use 8-bit T1 fonts
\usepackage{lmodern}        % https://github.com/rstudio/rticles/issues/343
\usepackage{hyperref}       % hyperlinks
\usepackage{url}            % simple URL typesetting
\usepackage{booktabs}       % professional-quality tables
\usepackage{amsfonts}       % blackboard math symbols
\usepackage{nicefrac}       % compact symbols for 1/2, etc.
\usepackage{microtype}      % microtypography
\usepackage{graphicx}

\title{They're Clutching up! Team Momentum in Round-Based Esports}

\author{
    Tony ElHabr
   \\
     \\
   \\
  \texttt{\href{mailto:anthonyelhabr@gmail.com}{\nolinkurl{anthonyelhabr@gmail.com}}} \\
  }


% tightlist command for lists without linebreak
\providecommand{\tightlist}{%
  \setlength{\itemsep}{0pt}\setlength{\parskip}{0pt}}


% Pandoc citation processing
\newlength{\cslhangindent}
\setlength{\cslhangindent}{1.5em}
\newlength{\csllabelwidth}
\setlength{\csllabelwidth}{3em}
\newlength{\cslentryspacingunit} % times entry-spacing
\setlength{\cslentryspacingunit}{\parskip}
% for Pandoc 2.8 to 2.10.1
\newenvironment{cslreferences}%
  {}%
  {\par}
% For Pandoc 2.11+
\newenvironment{CSLReferences}[2] % #1 hanging-ident, #2 entry spacing
 {% don't indent paragraphs
  \setlength{\parindent}{0pt}
  % turn on hanging indent if param 1 is 1
  \ifodd #1
  \let\oldpar\par
  \def\par{\hangindent=\cslhangindent\oldpar}
  \fi
  % set entry spacing
  \setlength{\parskip}{#2\cslentryspacingunit}
 }%
 {}
\usepackage{calc}
\newcommand{\CSLBlock}[1]{#1\hfill\break}
\newcommand{\CSLLeftMargin}[1]{\parbox[t]{\csllabelwidth}{#1}}
\newcommand{\CSLRightInline}[1]{\parbox[t]{\linewidth - \csllabelwidth}{#1}\break}
\newcommand{\CSLIndent}[1]{\hspace{\cslhangindent}#1}

\usepackage{longtable}
\usepackage{amsmath}
\begin{document}
\maketitle


\begin{abstract}
My research investigates patterns in round win percentages in
professional Search and Destroy (SnD) matches of the popular
first-person shooter game Call of Duty (CoD). First, I find evidence
supprting the hypothesis that round win probability can be modeled as a
constant across the series, although not at the naive 50\%. Second, I
examine post-streak round win probability, in search of evidence
positive recency (the ``hot hand'' fallacy) or negative recency (the
``gambler's falacy''). I find that teams perform significantly worse
than expected after streaks of 2, 3, and 4 wins when series end up going
to 9, 10, or 11 (maximum) rounds, suggesting the presence of negative
recency.
\end{abstract}


\hypertarget{introduction}{%
\section{Introduction}\label{introduction}}

\hypertarget{description-of-call-of-duty-search-and-destroy}{%
\subsection{Description of Call of Duty Search and
Destroy}\label{description-of-call-of-duty-search-and-destroy}}

Call of Duty (CoD), first released in 2003, is one of the most popular
first-person shooter (FPS) video game franchises of all-time. The most
popular mode in the competitive scene is ``Search and Destroy''
(SnD).\footnote{SnD bears resemblance to ``Bomb Defusal'' in
  Counter-Strike and ``Plant/Defuse'' in Valorant, two other FPS games
  played in more popular professional leagues.} SnD is a one-sided game
mode in which one team, the offensive side, tries to destroy one of two
designated bomb sites on the map.

In professional CoD SnD, a team take turns playing offense and defense
every round. They must win six rounds to win the series.\footnote{A
  maximum of 11 even rounds can be played. There is no ``sudden death''
  or ``win by two'' rule like there are for SnD equivalent in
  professional Counter-Strike and Valorant matches.} A round can end in
one of five ways:

\begin{enumerate}
\def\labelenumi{\arabic{enumi}.}
\tightlist
\item
  One team eliminates all members of the other team prior to a bomb
  plant. (Eliminating team wins.)
\item
  The offensive team eliminates all members of the defensive team after
  a bomb plant.\footnote{\begin{itemize}
    \tightlist
    \item
      The bomb can be picked up by any member of the offensive team.
    \item
      The bomb carrier is not obstructed at all by carrying the bomb
      (i.e.~movement is the same, weapon usage is the same).
    \item
      The defense does not get any visual indication for who is carrying
      the bomb.
    \item
      A bomb plant takes five seconds. The timer resets if the player
      stops planting site prior to completing it.
    \item
      A bomb defuse takes seven seconds. The timer resets if the player
      ``drops'' the bomb.
    \item
      The bomb takes 45 seconds to defuse after being planted.
    \end{itemize}} (Offense wins.)
\item
  The defensive team defuses the bomb after a bomb plant.\footnote{Often
    the defensive team will try to eliminate all team members prior to
    making the defuse, but in some cases, they may try to ``ninja''
    defuse.} (Defense wins.)
\item
  The offensive team does not make a plant by the time the round timer
  ends. (Defense wins.)
\end{enumerate}

I adopt the terminology ``series'' to refer to what CoD SnD players
typically call a ``match'', so as to emulate the terminology of playoff
series in professional leagues like the National Basketball Association,
National Hockey League, and Major League Baseball. A ``game'' or a
``match'' in such leagues is analogous to a ``round'' of CoD SnD.

\hypertarget{data}{%
\subsection{Data}\label{data}}

CoD has roughly gone through three eras of professional gaming: (1)
Major League Gaming (MLG) tournaments prior to 2016; (2) the CoD World
League (CWL), initiated in 2016; and (3) the 12-franchise CoD League
(CDL), operating since 2020. The CDL has completed three year-long
``seasons'' as of August 2022.\footnote{CoD is fairly unique compared to
  other esports in that it runs on an annual lifecycle (released coming
  in the late fall), where a new game is published every year under the
  same title. Each new game bears resemblance to past ones, often
  introducing relatively small variations (``improvements'') to
  graphics, game modes, and other facets of gameplay. During the CDL
  era, the games released have been Modern Warfare (2020), Cold War
  (2021) and Vanguard (2022).}

The data set consists of all SnD matches played in tournanaments and
qualifiers during the CDL era, totaling 7,792 rounds across 852 series.
Data was collected in spreadsheets by community member
``IOUTurtle''.\footnote{Data: \url{https://linktr.ee/CDLArchive}.
  Author: \url{https://twitter.com/IOUTurtle}}

The empirical offensive round win percentage across all rounds is
47.8\%.\footnote{Offensive round win percentage has been nearly constant
  across the three games during the CDL era: 1. 47.2\% in MW (2020) 2.
  47.9\% in Cold War (2021) 3. 48.1\% in Vanguard (2022)} Table
\ref{tbl:cod-o-win-prop-by-series-state} shows round win percentages by
series ``state'' (i.e.~the number of round wins by each team prior to an
upcoming round). Offensive round win rate is not quite constant,
although never veers more than 10\% from this global average.

\begin{longtable}{crrrrrr}
\caption{Offensive round win rates for the upcoming round, given both the offensive and defensive team's prior number of round wins}\label{tbl:cod-o-win-prop-by-series-state} \\
\toprule
& \multicolumn{6}{c}{Offense round wins} \\ 
\cmidrule(lr){2-7}
Defense round wins & 0 & 1 & 2 & 3 & 4 & 5 \\ 
\midrule
0 & 47.8\%
(852) & 46.6\%
(408) & 43.1\%
(216) & 43.5\%
(115) & 43.3\%
(67) & 40.5\%
(37) \\ 
1 & 48.6\%
(444) & 49.3\%
(418) & 51.5\%
(309) & 43.4\%
(205) & 43.3\%
(120) & 39.4\%
(99) \\ 
2 & 52.8\%
(218) & 48.9\%
(305) & 48.9\%
(315) & 46.6\%
(262) & 48.7\%
(189) & 42.1\%
(133) \\ 
3 & 54.5\%
(123) & 46.0\%
(200) & 49.6\%
(250) & 45.6\%
(248) & 44.4\%
(214) & 44.8\%
(174) \\ 
4 & 56.9\%
(65) & 54.5\%
(145) & 47.2\%
(193) & 44.7\%
(228) & 55.2\%
(221) & 50.5\%
(208) \\ 
5 & 47.4\%
(38) & 49.4\%
(83) & 47.1\%
(136) & 50.9\%
(175) & 45.2\%
(177) & 46.0\%
(202) \\ 
\bottomrule
\end{longtable}

\hypertarget{literature-review}{%
\section{Literature review}\label{literature-review}}

There have been a handful of studies of the distribution of games played
in a series of a professional sport. Most assume a constant probability
\(p\) of a given team winning a game in the series, regardless of the
series state. Mosteller (1952) observed that the American League had
dominated the National League in Major League Baseball's (MLB) World
Series matchups, implying that matchups should not modeled with
\(p = 0.5\). Mosteller proposed three approaches for identifying the
optimal constant probability value of the stronger team in the World
Series, finding \(p \approx 0.65\). in each case: (1) solving for \(p\)
from the empirical average number of games won by the loser of the
series, which he called the ``method of moments'' approach; (2)
maximizing the likelihood that the sample would have been drawn from a
population in which the probability of a team winning a game is constant
across the series (i.e.~maximum likelihood), and (3) minimizing the
chi-square goodness of fit statistic for \(p\).

Chance (2020) re-examines the constant probability notion in Major
League Baseball's World Series (1923--2018), the National Basketball
Association's Finals (1951--2018), and the National Hockey League's
Stanley Cup (1939--2018). Chance finds strong evidence against the null
hypothesis of \(p = 0.5\) in the MLB and NHL championship series when
applying Mosteller's first and second methods.\footnote{Chance goes on
  to outline a conditional probability framework (likelihood of winning
  a game given the series state) which can exactly explain the
  distribution of the number of games played.} Chance's work is closely
related to mine, and, in fact, provides a guide for the first part of my
investigation.

Momentum, one of most discussed topics in sports analytics, goes
hand-in-hand with a discussion of the nature of series
outcomes.\footnote{We often use use ``streaks'' and momentum
  interchangeably, but as (Steeger, Dulin, and Gonzalez 2021) note,
  momentum implies dependence between events, whereas streaking does
  not.} Two opposing fallacies are observed in the context of momentum:
the ``gambler's fallacy'' (negative recency) and ``hot hand fallacy''
(positive recency). Per Ayton et al. (2004), negative recency is ``the
belief that, for random events, runs of a particular outcome \ldots{}
will be balanced by a tendency for the opposite outcome'', while
positive recency is the expectation of observing future results that
match recent results.

Studying both player streaks and team streaks in basketball, in both
observational and controlled settings. Gilovich et al. (1985) do not
find evidence for the hot hand phenomenon. Recently, Miller et al.
(2018) refuted the conclusions of Gilovich et al.~, finding mathematical
evidence that seems to support negative recency. Specifically, they find
that a ``bias exists in a common measure of the conditional dependence
of present outcomes on streaks of past outcomes in sequential data''
(streak selection bias) that imply that, under i.i.d. conditions, ``the
proportion of successes among the outcomes that immediately follow a
streak of consecutive successes is expected to be strictly less than the
underlying (conditional) probability of success''. I agree with Miller's
findings, accounting for the streak selection bias in my study of
momentum.

Despite the plethora of existing research on games played in a series
and momentum in sports, these topics have yet to be investigated heavily
in esports. Work has been done to examine in-round win probability in
other FPS titles such as Counter-Strike (Xenopoulos, Freeman, and Silva
2022) and Valorant (DeRover 2021), both of which are round-based like
CoD SnD. However, research on round-level trends is sparse, perhaps
because games like Counter-Strike and Valorant both have economic
aspects that can create clear advantages on side in a given round, given
how prior rounds played out.\footnote{Additionally, both Counter-Strike
  and Valorant have overtime rules and blocked offensive/defensive roles
  (i.e.~playing either offense or defense for many consecutive rounds).}

\hypertarget{methodology-results-and-discussion}{%
\section{Methodology, results, and
discussion}\label{methodology-results-and-discussion}}

First, I investigate the constant probability assumption and the
distribution of rounds played in a series. Afterwards, I investigate
momentum, building on my learnings from the constant probability
assumption analysis.

\hypertarget{sec:analysis-1}{%
\subsection{Distribution of rounds played}\label{sec:analysis-1}}

The general formula for the probability \(P_E(i)\) that a best-of-\(s\)
series lasts \(i\) rounds given constant probability \(p\) of one
team\footnote{If \(p > 0.5\), then I might say that this team is the
  better team (known omnisciently).} winning each round is

\[
q = 1 - p, s_1 = \frac{s_1 - 1}{2}, s_2 = \frac{s_1 + 1}{2}, s_2^{'} = 1 - s_2
\]

\begin{equation}\protect\hypertarget{eq-series-length}{}{
P_E(i) = \frac{(i - 1)!}{s_1!(i - s_2)!}(p^{s_2}q^{s_2^{'}} + p^{s_2^{'}}q^{s_2}).
}\label{eq:series-length}\end{equation}

Table \ref{tbl:cod-prob-series-lasting-i-rounds} shows the expected
probabilities for \(s = 11\) under the naive assumption that
\(p_0 = 0.5\) for a given round, along with the observed proportions
\(P_O(i)\) in CoD SnD.

\begin{longtable}{rrrr}
\caption{The probabilities that a best-of-11 series lasts $i$ rounds ($P_E(i)$, where $i \in [6, 7, 8, 9, 10, 11]$) under the assumption that each team has a 50\% probability ($p_0 = 0.5$) of winning each game. The observed frequencies observed are shown as a count $N_O(i)$ and as a proportion $P_O(i)$ of all series.}\label{tbl:cod-prob-series-lasting-i-rounds} \\
\toprule
&  & \multicolumn{2}{c}{Observed} \\ 
\cmidrule(lr){3-4}
Series lasts $i$ rounds & $P_E(i)$ & $P_O(i)$ & $N_O(i)$ \\ 
\midrule
6 & $3.1\%$ & $4.7\%$ & 40 \\ 
7 & $9.4\%$ & $11.9\%$ & 101 \\ 
8 & $16.4\%$ & $16.5\%$ & 141 \\ 
9 & $21.9\%$ & $21.7\%$ & 185 \\ 
10 & $24.6\%$ & $21.5\%$ & 183 \\ 
11 & $24.6\%$ & $23.7\%$ & 202 \\ 
\bottomrule
\end{longtable}

Calculating the chi-square goodness of fit statistic

\begin{equation}\protect\hypertarget{eq:series-length}{}{
\chi^2 = \sum^{11}_{i=6} \frac{(P_O(i) - P_E(i))^2}{P_E(i)}, i \in R = [6, 7, 8, 9, 10, 11].
}\label{eq:chi-squ-test}\end{equation}

as 16.0 (p-value of 0.0068), I can comfortably reject the constant
probability null hypothesis for \(p_0 = 0.5\), event at a confidence
level of \(\alpha = 0.01\).

Table \ref{tbl:mosteller-methods-results} shows the alternate values for
the constant round win probability \(p\) that I find when applying the
three methods suggested by Mosteller (1952). Each is approximately or
equal to 0.575, and, when applying Equation \ref{eq:chi-squ-test}, each
results in a \(\chi^2\) value for which I cannot reject the constant
probability null hypothesis.

\begin{longtable}[]{@{}lrr@{}}
\caption{Alternate estimates of the constant probability \(p\) for winning a given round in a CoD SnD, applying the three methods suggested by Mosteller (1952), in addition to the naive \(p = 0.5\).}\label{tbl:mosteller-methods-results} \\
\toprule()
Method & \(p\) & \(\chi^2\) (p-value) \\
\midrule()
\endfirsthead
\toprule()
Method & \(p\) & \(\chi^2\) (p-value) \\
\midrule()
\endhead
0. Naive & 0.5000 & 16.0 (\textless=0.01) \\
1. Method of moments & 0.5725 & 3.6 (0.6) \\
2. Maximum likelihood & 0.5750 & 3.5 (0.62) \\
3. Minimum \(\chi^2\) & 0.5775 & 3.5 (0.62) \\
\bottomrule()
\end{longtable}

Table \ref{tbl:expected-series-lengths-alternative-ps} shows the new
\(P_E(i)\) when re-applying Equation \ref{eq:series-length} for the
maximum likelihood estimate \(p_2 = 0.575\).\footnote{The method of
  moments and minimum \(\chi^2\) estimates for \(p\) are omitted simply
  because the results would be nearly identical to those for the maximum
  likelihood estimate of \(p\) (since they are all \(\approx 0.575\).} I
observe that \(P_E(i)\) is notably larger for \(p_{2}\) when
\(i \in [6, 7]\), more closely matching \(P_O(i)\). \(p_{2}\) is also
closer to \(P_O(i)\) for \(i \in [9, 10]\), although not for
\(i \in [8, 11]\).

\begin{longtable}[]{@{}rrrr@{}}
\caption{The probabilities that a best-of-11 series lasts $i$ rounds under the naive assumption $p_0 = 0.5$ and the maximum likelihood estimation $p_2 = 0.575$.}\label{tbl:expected-series-lengths-alternative-ps} \\
\toprule()
Series lasts \(i\) rounds & \(p_0\) = 0.5 & \(p_2\) = 0.575 & \(P_O(i)\) \\
\midrule()
\endhead
6 & 3.1\% & 4.2\% & 4.7\% \\
7 & 9.4\% & 11.2\% & 11.9\% \\
8 & 16.4\% & 17.8\% & 16.5\% \\
9 & 21.9\% & 21.8\% & 21.7\% \\
10 & 24.6\% & 23.0\% & 21.5\% \\
11 & 24.6\% & 22.0\% & 23.7\% \\
\bottomrule()
\end{longtable}

It is fair to conclude that the constant round win probability
assumption can be valid in CoD SnD series with the appropriate choice of
\(p\) (\(\approx 0.575\)).

\hypertarget{sec:analysis-2}{%
\subsection{Momentum}\label{sec:analysis-2}}

Despite the implication that one can model team round win probability in
CoD SnD as a constant, anecdotally many believe in momentum i.e.~streaks
not due to random chance. I now evaluate win frequency in rounds
following a streak of length \(k\). I account for the length of the
series (\(i\) rounds), modeling the expected round win probability after
a streak of length \(k\) as

\begin{equation}\protect\hypertarget{eq:pki}{}{
  \hat{P}(\text{win round | k-round winning streak, series lasting i rounds})=\{
  \begin{array}{cl}
    \frac{6 - k}{i - k}, & \text{team wins series} \\
    \frac{11 - 6 - k}{i - k}, & \text{team loses series}
  \end{array}.
}\label{eq:pki}\end{equation}

Miller et al. (2018) provide evidence for negative recency suggesting
that

\begin{equation}\protect\hypertarget{eq:pkid}{}{
  \begin{array}{rcl}
  \hat{P}_{ki,\text{diff}} & = & \hat{P}(\text{win round | k-round winning streak, series lasting i rounds}) -  \\
    & & \hat{P}(\text{lose round | k-round losing streak, series lasting i rounds}) \\
    E[\hat{P}_{ki,\text{diff}}] & \neq & 0
  \end{array}
}\label{eq:pkid}\end{equation}

given a deterministic i.i.d. Bernoulli probability
\(\hat{p}(\text{win round | k-round winning streak, series lasting i rounds)}\)
(Equation \ref{eq:pki}.\footnote{Note that these probabilites are
  separate from those in Section \textbackslash ref\{sec:analysis-1\}.}

Given that people typically perceive streaks as beginning after the
third success (or failure) at minimum (Carlson and Shu 2007), I focus on
round win percentages in rounds immediately following streaks of three
wins, \(k = 3\).\footnote{Three happens to also be a reasonable number
  for series that lasts at maximum 11 rounds.}

\hypertarget{references}{%
\section*{References}\label{references}}
\addcontentsline{toc}{section}{References}

\hypertarget{refs}{}
\begin{CSLReferences}{1}{0}
\leavevmode\vadjust pre{\hypertarget{ref-ayton2004}{}}%
Ayton, Peter, and Ilan Fischer. 2004. {``The Hot Hand Fallacy and the
Gambler{'}s Fallacy: Two Faces of Subjective Randomness?''} \emph{Memory
\& Cognition} 32 (8): 13691378.

\leavevmode\vadjust pre{\hypertarget{ref-carlson2007}{}}%
Carlson, Kurt A., and Suzanne B. Shu. 2007. {``The Rule of Three: How
the Third Event Signals the Emergence of a Streak.''}
\emph{Organizational Behavior and Human Decision Processes} 104 (1):
113121.

\leavevmode\vadjust pre{\hypertarget{ref-chance2020}{}}%
Chance, Don. 2020. {``Conditional Probability and the Length of a
Championship Series in Baseball, Basketball, and Hockey.''}
\emph{Journal of Sports Analytics} 6 (2): 111--27.
\url{https://doi.org/10.3233/JSA-200422}.

\leavevmode\vadjust pre{\hypertarget{ref-derover2021}{}}%
DeRover, DeMars. 2021. {``Round Win Probabilities Based on Who's Alive +
Time.''}
\url{https://www.reddit.com/r/VALORANT/comments/n3lpoo/round_win_probabilities_based_on_whos_alive_time/}.

\leavevmode\vadjust pre{\hypertarget{ref-gilovich1985}{}}%
Gilovich, Thomas, Robert Vallone, and Amos Tversky. 1985. {``The Hot
Hand in Basketball: On the Misperception of Random Sequences.''}
\emph{Cognitive Psychology} 17 (3): 295314.

\leavevmode\vadjust pre{\hypertarget{ref-miller2018}{}}%
Miller, Joshua B., and Adam Sanjurjo. 2018. {``Surprised by the Hot Hand
Fallacy? A Truth in the Law of Small Numbers.''} \emph{Econometrica} 86
(6): 20192047.

\leavevmode\vadjust pre{\hypertarget{ref-mosteller1952}{}}%
Mosteller, Frederick. 1952. {``The World Series Competition.''}
\emph{Journal of the American Statistical Association} 47 (259): 355380.

\leavevmode\vadjust pre{\hypertarget{ref-steeger2021}{}}%
Steeger, Gregory M., Johnathon L. Dulin, and Gerardo O. Gonzalez. 2021.
{``Winning and Losing Streaks in the National Hockey League: Are Teams
Experiencing Momentum or Are Games a Sequence of Random Events?''}
\emph{Journal of Quantitative Analysis in Sports} 17 (3): 155170.

\leavevmode\vadjust pre{\hypertarget{ref-xenopoulos2022}{}}%
Xenopoulos, Peter, William Robert Freeman, and Claudio Silva. 2022.
{``Analyzing the Differences Between Professional and Amateur Esports
Through Win Probability.''} In, 34183427.

\end{CSLReferences}

\bibliographystyle{unsrt}
\bibliography{references.bib}


\end{document}
