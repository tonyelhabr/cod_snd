\documentclass{article}

\usepackage{arxiv}

\usepackage[utf8]{inputenc} % allow utf-8 input
\usepackage[T1]{fontenc}    % use 8-bit T1 fonts
\usepackage{lmodern}        % https://github.com/rstudio/rticles/issues/343
\usepackage{hyperref}       % hyperlinks
\usepackage{url}            % simple URL typesetting
\usepackage{booktabs}       % professional-quality tables
\usepackage{amsfonts}       % blackboard math symbols
\usepackage{nicefrac}       % compact symbols for 1/2, etc.
\usepackage{microtype}      % microtypography
\usepackage{graphicx}

\title{The Hot Hand Fallacy in Call of Duty Search and Destroy}

\author{
    Person One
   \\
     \\
   \\
  \texttt{\href{mailto:personone@domain.com}{\nolinkurl{personone@domain.com}}} \\
   \And
    Person Two
   \\
     \\
   \\
  \texttt{\href{mailto:persontwo@domain.com}{\nolinkurl{persontwo@domain.com}}} \\
  }


% tightlist command for lists without linebreak
\providecommand{\tightlist}{%
  \setlength{\itemsep}{0pt}\setlength{\parskip}{0pt}}


% Pandoc citation processing
\newlength{\cslhangindent}
\setlength{\cslhangindent}{1.5em}
\newlength{\csllabelwidth}
\setlength{\csllabelwidth}{3em}
\newlength{\cslentryspacingunit} % times entry-spacing
\setlength{\cslentryspacingunit}{\parskip}
% for Pandoc 2.8 to 2.10.1
\newenvironment{cslreferences}%
  {}%
  {\par}
% For Pandoc 2.11+
\newenvironment{CSLReferences}[2] % #1 hanging-ident, #2 entry spacing
 {% don't indent paragraphs
  \setlength{\parindent}{0pt}
  % turn on hanging indent if param 1 is 1
  \ifodd #1
  \let\oldpar\par
  \def\par{\hangindent=\cslhangindent\oldpar}
  \fi
  % set entry spacing
  \setlength{\parskip}{#2\cslentryspacingunit}
 }%
 {}
\usepackage{calc}
\newcommand{\CSLBlock}[1]{#1\hfill\break}
\newcommand{\CSLLeftMargin}[1]{\parbox[t]{\csllabelwidth}{#1}}
\newcommand{\CSLRightInline}[1]{\parbox[t]{\linewidth - \csllabelwidth}{#1}\break}
\newcommand{\CSLIndent}[1]{\hspace{\cslhangindent}#1}

\usepackage{amsmath}
\begin{document}
\maketitle


\begin{abstract}
Our research investigates patterns in round win percentages in
professional Search and Destroy (SnD) matches of the popular
first-person shooter game Call of Duty (CoD). First, we find evidence
supporting the hypothesis that round win probability can be modeled as a
constant across rounds in the series, although not at the naive 50\%.
Second, we examine the proportion of round wins immediately following a
win streak. Given streak length, series length, and series winner, we
find no evidence that post-streak win proportion is significantly
different than expected, suggesting that the perception of momentum at
the team level is deceiving. Wald-Wolfowitz run tests also fail to
provide evidence for the hot hand phenomenon.
\end{abstract}

\keywords{
    esports
   \and
    Call of Duty
   \and
    hot hand
   \and
    runs test
  }

\hypertarget{introduction}{%
\section{Introduction}\label{introduction}}

\hypertarget{call-of-duty-search-and-destroy}{%
\subsection{Call of Duty Search and
Destroy}\label{call-of-duty-search-and-destroy}}

Call of Duty (CoD), first released in 2003, is one of the most popular
first-person shooter (FPS) video game franchises of all-time. The most
popular mode in the competitive scene is ``Search and Destroy'' (SnD),
in which one team tries to destroy one of two designated bomb sites on
the map.\footnote{SnD bears resemblance to ``Bomb Defusal'' in
  Counter-Strike and ``Plant/Defuse'' in Valorant, two other FPS games
  played in more popular professional leagues.}

In professional CoD SnD, the two teams\footnote{In the 2020 season,
  teams played with five players each; in the 2021 and 2022 seasons,
  teams played with four players each.} take turns playing offense and
defense every round. The first to six round wins (best-of-11 round
format) is declared the series winner.\footnote{A short list of rules
  that govern timing follows.} A round can end in one of four ways:

\begin{itemize}
\tightlist
\item
  Each round has a 90 second time limit, not counting the potential
  extension granted if a bomb is planted
\item
  A bomb plant takes five seconds. The timer resets if the player stops
  planting site prior to completing it.
\item
  A bomb defuse takes seven seconds. The timer resets if the player
  ``drops'' the bomb.
\item
  Once the bomb is planted, the round timer is set to 45 seconds.
\end{itemize}

\begin{enumerate}
\def\labelenumi{\arabic{enumi}.}
\tightlist
\item
  One team eliminates all members of the other team prior to a bomb
  plant. (Eliminating team wins.)
\item
  The offensive team eliminates all members of the defensive team after
  a bomb plant. (Offense wins.)
\item
  The defense defuses the bomb after a bomb plant. (Defense wins.)
\item
  The offense does not make a plant by the time the round timer ends.
  (Defense wins.)
\end{enumerate}

We adopt the terminology ``series'' to refer to a single best-of-11
matchup, so as to mirror the terminology of playoff series in
professional leagues like the National Basketball Association (NBA),
National Hockey League (NHL), and Major League Baseball (MLB). A
``game'' or a ``match'' in such leagues is analogous to a ``round'' of
CoD SnD for the purposes of this paper. Note that SnD is not the only
game mode played in competitive CoD.\footnote{Hardpoint has been played
  as the first and fourth game modes in a matchup. The third game mode
  was Domination in the 2020 season, and Control in the 2021 and 2022
  seasons.} Teams play in a head-to-head, best-of-five format, where SnD
is always played as the second and fifth game modes. The best-of-five
matchup could also be called a ``series'', but since we analyze only the
SnD games, we refer to the SnD games as series.

\hypertarget{data}{%
\subsection{Data}\label{data}}

The data set consists of all SnD matches played in major tournaments and
qualifiers during the past three years\footnote{Since the release of
  ``Call of Duty: Modern Warfare'' in fall 2019, professional CoD has
  been orchestrated by the 12-franchise CoD League (CDL). The CDL has
  completed three year-long ``seasons'' as of August 2022.}, totaling
7,792 rounds across 852 series.\footnote{288 of the series occur in
  major tournaments, which are considered to be more competitive than
  the qualifiers since they are played in person (COVID-permitting),
  eliminating randomness due to online latency and ping. The qualifiers
  are played online.} Data has been collected in spreadsheets by
community member ``IOUTurtle''.\footnote{Raw data:
  \url{https://linktr.ee/CDLArchive}. Cleaned data:
  \url{https://github.com/\%7Bauthor\%7D/\%7Brepo\%7D/blob/master/data/\%7Bfile\%7D.csv}.}

CoD is fairly unique compared to other esports in that it runs on an
annual lifecycle, with releases coming in the late fall. A new game
under the same brand---``Call of Duty''---is published every year by a
rotating set of developers. Each new game bears resemblance to past
ones, often introducing relatively small variations (``improvements'')
to graphics, game modes, and other facets of gameplay. During the CDL
era, the games released have been ``Modern Warfare'' (2020 season),
``Cold War'' (2021 season) and ``Vanguard'' (2022 season).

Sweeps (6-0 series) make up 4.7\% of all series, while 6-5 series make
up 23.7\% of series. The observed offensive round win
percentage\footnote{We use the terms ``percentage'', ``probability'',
  and ``proportion'' synonymously throughout this paper when discussing
  round win rates. For the most part, we restrict ``percentage'' usage
  to observed values and ``probability'' usage to expected values, and
  we use ``proportion'' to refer to either observed or expected values.}
across all rounds is \(\bar{\tau}\) = 47.8\%.\footnote{Offensive round
  win percentage has been nearly constant across the three games during
  the CDL era: 47.2\% in Modern Warfare; 47.9\% in Cold War; and 48.1\%
  in Vanguard} Table \ref{tbl:o-win-prop-by-series-state} shows
offensive round win percentages by series ``state'', i.e.~the number of
round wins by each team prior to an upcoming round. Offensive round win
percentage is not quite constant, although never veers more than 10\%
from \(\bar{\tau}\).

\begin{table}

\caption{Offensive round win percentage (\%) for the upcoming round, given both the offensive and defensive team's prior number of round wins. Numbers in parentheses indicate sample sizes.}

\centering
\begin{tabular}{crrrrrr}
\toprule
& \multicolumn{6}{c}{Offense round wins} \\ 
\cmidrule(lr){2-7}
Defense round wins & 0 & 1 & 2 & 3 & 4 & 5 \\ 
\midrule

0 & 47.8 (852) & 46.6 (408) & 43.1 (216) & 43.5 (115) & 43.3 (67)  & 40.5 (37)  \\
1 & 48.6 (444) & 49.3 (418) & 51.5 (309) & 43.4 (205) & 43.3 (120) & 39.4 (99)  \\
2 & 52.8 (218) & 48.9 (305) & 48.9 (315) & 46.6 (262) & 48.7 (189) & 42.1 (133) \\
3 & 54.5 (123) & 46.0 (200) & 49.6 (250) & 45.6 (248) & 44.4 (214) & 44.8 (174) \\
4 & 56.9 (65)  & 54.5 (145) & 47.2 (193) & 44.7 (228) & 55.2 (221) & 50.5 (208) \\
5 & 47.4 (38)  & 49.4 (83)  & 47.1 (136) & 50.9 (175) & 45.2 (177) & 46.0 (202) \\

\bottomrule
\end{tabular}

\label{tbl:o-win-prop-by-series-state}

\end{table}

The professional competition format balances the frequency with which
teams play offense or defense to start a series, to the extent possible.
Thus, win percentages in certain states are subjected to minimal
selection bias, i.e.~bias due to better teams playing defense more
frequently.\footnote{At tournaments, higher-seeded (``better'') teams
  choose whether they want to start on offense in the SnD series, played
  as the second or fifth game mode of the best-of-five matchup. Their
  opponent is assigned to start on offense in the SnD slot not chosen.
  Anecdotally, the better team tends to choose to play defense in the
  first round of the SnD series played as the second game mode in the
  best-of-five matchup, although this choice is not consistent across
  teams, or even with the same team over time.}

Outside of tournaments, teams play each other twice throughout the
season, constituting qualifiers. Each team gets to play the
higher-seeded role once in their head-to-head matchups with a given
team.

This format has minimized the range of rates at which teams start series
playing offense. The highest rate at which a team has started their
series playing defense across all tournament and qualifier SnD series in
a single season is 66\% (31 of 47). The lowest is 38\% (16 of 42).

\hypertarget{literature-review}{%
\section{Literature review}\label{literature-review}}

There have been a handful of studies of the distribution of games played
in a series of a professional sport. Most assume a constant probability
\(\phi\) of a given team winning a game in the series, regardless of the
series state. Mosteller (\protect\hyperlink{ref-mosteller1952}{1952})
observed that the American League had dominated the National League in
MLB World Series matchups through 1952, implying that games should not
be modeled with a constant \(\phi_0 = 0.5\). Mosteller proposed three
approaches for identifying the optimal constant probability value of the
stronger team in the World Series, finding \(\hat{\phi} \approx 0.65\)
in each case: (1) solve for \(\phi\) from the observed average number of
games won by the loser of the series, which he called the ``method of
moments'' approach; (2) maximize the likelihood that the sample would
have been drawn from a population in which the probability of a team
winning a game is constant across the series (i.e.~maximum likelihood),
and (3) minimize the chi-squared goodness of fit statistic \(\chi^2\) as
a function of \(\phi\).

Chance (\protect\hyperlink{ref-chance2020}{2020}) re-examined the
constant probability notion in the MLB World Series (1923--2018), the
NBA Finals (1951--2018), and the NHL Stanley Cup (1939--2018). Chance
found strong evidence against the null hypothesis of \(\phi_0 = 0.5\) in
the MLB and NHL championship series when applying Mosteller's first and
second methods.

The concept of momentum goes hand-in-hand with a discussion of the
nature of series outcomes.\footnote{We often use ``streaks'' and
  momentum interchangeably, but as Steeger et al.
  (\protect\hyperlink{ref-steeger2021}{2021}) note, momentum implies
  dependence between events, whereas streaking does not.} Two opposing
fallacies are observed in the context of momentum: the ``gambler's
fallacy'' (negative recency) and the ``hot hand fallacy'' (positive
recency). Per Ayton et al. (\protect\hyperlink{ref-ayton2004}{2004}),
negative recency is ``the belief that, for random events, runs of a
particular outcome \ldots{} will be balanced by a tendency for the
opposite outcome'', while positive recency is the expectation of
observing future results that match recent results.

Studying both player streaks and team streaks in basketball, in both
observational and controlled settings, Gilovich et al.
(\protect\hyperlink{ref-gilovich1985}{1985}), henceforth GVT, did not
find evidence for the hot hand phenomenon. However, Miller and Sanjurjo
(\protect\hyperlink{ref-miller2018}{2018}), henceforth MS, provided a
framework for quantifying streak selection bias, which effectively works
in the manner posited by the gambler's fallacy. Specifically, MS stated
that a ``bias exists in a common measure of the conditional dependence
of present outcomes on streaks of past outcomes in sequential data''
implying that, under i.i.d. conditions, ``the proportion of successes
among the outcomes that immediately follow a streak of consecutive
successes is expected to be strictly less than the underlying
(conditional) probability of success''. When applying streak selection
bias to GVT's data, MS came to the opposite conclusions as GVT.

Other research has borrowed techniques from the field of quality
control, such as identifying unlikely sequences of events with the
Wald-Wolfowitz runs test (Peel and Clauset 2015, Steeger et al.~2021).
Peel and Clauset found no evidence for unlikely sequences of scoring
events in the NHL, College Football, and National Football League,
although did in the NBA. As a check on their entropy approach to
momentum identification, Steeger et al.~found several NHL teams with
sequences of wins in the 2018-2019 regular season that violated the
Wald-Wolfowitz null hypothesis.

\hypertarget{our-contribution}{%
\subsection{Our contribution}\label{our-contribution}}

Despite the plethora of existing research on games played in a series
and momentum in sports, these topics have yet to be investigated
heavily, if at all, in esports. Work has been done to examine
intra-round, second-by-second win probability in other FPS titles such
as Counter-Strike (\protect\hyperlink{ref-xenopoulos2022}{Xenopoulos,
Freeman, and Silva 2022}) and Valorant
(\protect\hyperlink{ref-derover2021}{DeRover 2021}), both of which are
round-based like CoD SnD. However, considering Counter-Strike and
Valorant specifically, research on round-level trends seems
non-existent, perhaps for one of several reasons:

\begin{enumerate}
\def\labelenumi{\arabic{enumi}.}
\tightlist
\item
  Both have economic aspects that can create clear advantages for one
  team in a round, given how prior rounds played out. CoD has no such
  equivalent, except for perhaps ``score streaks'', which infrequently
  occur.
\item
  Teams play either offense or defense for many consecutive rounds.
  (Counter-Strike has a 15-15-1 format and Valorant has a 12-12-1
  format.) On the other hand, teams in CoD SnD rotate attacking and
  defending roles every round, minimizing the effect of map asymmetry on
  streaks of wins.\footnote{Professional CoD has a set of maps (at least
    three, depending on the season) on which teams choose to play SnD.
    Most are asymmetric, introducing slight advantages for the offense
    or defense that do not exist on other maps. Counter-Strike and
    Valorant also have their own sets of maps for their analogous SnD
    modes.}
\item
  Both Counter-Strike and Valorant have overtime rules---a team must win
  by two rounds---which can make end-of-series sequences difficult to
  model properly. CoD SnD does not have overtime rules.
\end{enumerate}

To our knowledge, there is no existing public statistical research on
the CoD SnD format, beyond descriptive analysis on social media. While
intra-round trends may be more directly applicable to teams looking for
an advantage on their competition, broader investigation of a concept
like the hot-hand fallacy in a sport where it has not yet been
investigated---particularly one that is less subjected to factors that
may be difficult to control for, e.g.~weather---should be useful as an
inspiration for future researchers.

\hypertarget{methodology}{%
\section{Methodology}\label{methodology}}

\hypertarget{distribution-of-rounds-played}{%
\subsection{Distribution of rounds
played}\label{distribution-of-rounds-played}}

In a best-of-\(s\) format, assuming a constant round win probability
\(\phi\) (where \(0 \leq \phi \leq 1\)) for a given team, the expected
proportion of series ending in \(r\) rounds (where \(r \leq s\)) is
given by Equation \ref{eq:series-length}.

\begin{equation}\label{eq:m}
m = \frac{s + 1}{2}.
\end{equation}

\begin{equation}\label{eq:series-length}
\hat{\Phi}(r) = \frac{(r - 1)!}{(m - 1)!(s - r)!}(\phi^{m}(1 - \phi)^{r - m} + \phi^{r - m}(1 - \phi)^m).
\end{equation}

For example, assuming \(\phi = 0.5\), the probability of a series ending
in exactly nine rounds in CoD SnD, where \(s=11\) (implying \(m=6\)), is

\[
\hat{\Phi}(9) = \frac{(9 - 1)!}{(6 - 1)!(11 - 9)!}(0.5^{6}(1 - 0.5)^{9 - 6} + 0.5^{9 - 6}(1 - 0.5)^6) = 56 (0.5^9 + 0.5^9) = 0.21875.
\]

To evaluate the constant round win probability hypothesis, we can
compute the test statistic

\begin{equation}\label{eq:chi-squ}
\chi^2 = \sum_{r \in R} \frac{(\Phi(r) - \hat{\Phi}(r))^2}{\hat{\Phi}(r)}
\end{equation}

where \(\hat{\Phi}(r)\) and \(\Phi(r)\) represent the expected and
observed round win proportions respectively, and where
\(R = [6, 7, 8, 9, 10, 11]\) for CoD SnD. The null hypothesis is that
the test statistic is chi-squared distributed, where the number of
degrees of freedom is \(\vert R \vert -1 = 5\) for Cod SnD. We use a
significance level of \(\alpha = 0.05\) for this test and all others in
this paper.

\hypertarget{momentum}{%
\subsection{Momentum}\label{momentum}}

\hypertarget{ms-post-streak-probability}{%
\subsubsection{MS post-streak
probability}\label{ms-post-streak-probability}}

Let us now consider round win proportions immediately following a streak
of \(k\) wins, given that the series lasts \(r\) rounds. Removing our
knowledge of a streak, we might expect the win proportion for a round to
be

\begin{equation}\label{eq:pwr}
\hat{P}_0(\text{win} | r) = \hat{P}^{+|r}_0 := \begin{cases} 
\frac{m}{r} & \text{team wins series}, \\ 
\frac{r - m}{r} & \text{team loses series},
\end{cases}
\end{equation}

for \(r \in R\) and \(m\) from Equation \ref{eq:m}.

As implied by MS's Theorem 1, we should expect the proportion of round
wins immediately following a streak of \(k\) rounds wins for a series
ending in \(r\) rounds, \(\hat{P}^{+|k,r}_{MS}\), to be strictly less
than the observed proportion of round wins given the series length, but
not given knowledge of the potential existence of a streak, \(P^{+|r}\),
i.e.~the analogue of the expected proportion in Equation \ref{eq:pwr}
for the observed data. However, our context is fundamentally different
from that of MS, who evaluate data in controlled settings, unaffected by
opposition.\footnote{GVT also perform statistical tests on shots from
  players in live games, i.e.~``observational'' data, but they note that
  their findings are likely affected by player shot selection influenced
  by the opposing team's defensive strategy. This is similar to our
  setting.}

The number of trials is fixed in their experimental designs; in CoD SnD,
however, the number of rounds played is determined as a function of the
max number of possible rounds (\(s\)) and whether or not the team wins
the series. The single round win percentage in CoD SnD---analogous to a
player's shooting percentage in MS's analysis of basketball players---is
not independent of the opponent. Consequently, the i.i.d. assumption of
MS's theorem regarding post-streak success may or may not be valid.
Nonetheless, we still leverage \(\hat{P}^{+|k,r}_{MS}\) as a reference
for our results.

\hypertarget{notional-post-streak-probability}{%
\subsubsection{``Notional'' post-streak
probability}\label{notional-post-streak-probability}}

Let us consider another form of the expected proportion of rounds won
immediately after a streak of \(k\) round wins in a best-of-\(s\) series
ending in \(r\) rounds: the ``notional'' proportion

\begin{equation}\label{eq:pwkr}
\hat{P}_0(\text{win} | k, r) = \hat{P}^{+|k,r}_0 := \begin{cases}
\frac{m - k}{r - k} & \text{team wins series}, r > m; \\
1 & \text{team wins series}, r = m; \\
\frac{s - m - k}{r - k} & \text{team loses series}, r > m + k; \\
0 & \text{team loses series}, r = m + k,
\end{cases}
\end{equation}

where \(r \in R\) and \(m\) from Equation \ref{eq:m}.

We can perform a two-tailed binomial test to evaluate the null
hypothesis \(\omega = \omega_0\) for the observed proportion of
successes, \(\omega\), and a null proportion, \(\omega_0\) (where
\(0 \leq \omega_0 \leq 1\)). The probability of exactly \(r^+\)
successes (and \(r^- = r - r^+\) failures) in a sample of \(r\) trials
is

\begin{equation}\label{eq:binom}
\binom {r}{r^+} \omega^{r^+}(1-\omega)^{r^-},
\end{equation}

where \(\omega = \frac{r^+}{r}\) is the observed proportion of
successes. If we can reject the null hypothesis---treating \(P^{+|k,r}\)
and \(\hat{P}^{+|k,r}_0\) as \(\omega\) and \(\omega_0\)
respectively---then we might consider team momentum plausible. Since we
perform this test for many combinations of \(k\) and \(r\), we adjust
the \(p\)-values with the Benjamini and Yekutieli (BY) false discovery
procedure (\protect\hyperlink{ref-benjamini2001}{2001}).

As a reference, we can perform the same binomial test for MS's
proportion, \(\hat{P}^{+|k,r}_{MS}\), which adjusts for streak selection
bias. However, given the caveats mentioned before regarding the
application of MS's theorem to our setting, the results of such binomial
tests should be heeded with caution.

We can further decompose Equation \ref{eq:pwkr} by the round \(i\)
(where \(k < i \leq r\)) in which the streak of length \(k\) carries
into.

\begin{equation}\label{eq:pwkri}
\hat{P}_0(\text{win} | k, r, i) = \hat{P}^{+|k,r,i}_0 := \begin{cases}
\frac{m-k-1}{r-k-1} & i < r-1; \\
\frac{1}{2} & i = r-1; \\
1 & \text{team wins series}, i = r; \\
0 & \text{team loses series}, i = r. \\
\end{cases}
\end{equation}

Again, we can apply a binomial test to evaluate the hypothesis that the
expected proportion, \(\hat{P}^{+|k,r,i}_0\), is equal to the observed
proportion, \(P^{+|k,r,i}\).

\hypertarget{wald-wolfowitz-runs-test}{%
\subsubsection{Wald-Wolfowitz runs
test}\label{wald-wolfowitz-runs-test}}

We can attempt to detect the hot hand phenomenon more broadly by
focusing on non-random sequences identified by the Wald-Wolfowitz runs
test. Note that the probability of success and failure need not be
equal; thus, we can incorporate our findings regarding constant round
win probability, specifying that the probability of success is
\(\phi_2 = 0.575\).

Under the null hypothesis for the Wald-Wolfowitz runs test, the number
of runs in a sequence of \(r\) trials is a random variable that can take
on values \(+\) or \(-\) and arrive at \(r^+\) successes (and \(r^-\)
failures), with the following mean \(\mu_r\) \footnote{We calculate
  \(\mu\) and \(\sigma^2\) for each number of rounds \(r \in R\).} and
variance \(\sigma_r^2\):

\begin{equation}\label{eq:ww}
\mu_r = \frac{2r^{+}r^{-}}{r} + 1, \sigma_r^2 = \frac{(\mu_r-1)(\mu_r-2)}{r-1}.
\end{equation}

The null hypothesis is that the test statistic,

\begin{equation}\label{eq:wwz}
z_r = \frac{r^* - \mu_r}{\sigma_r},
\end{equation}

where \(r^*\) is the number of runs, is normally distributed.

We can subset the observed series sequences, \(\zeta_j\), to those that
violate the null hypothesis for the runs test and perform a test of
equal proportions, where the null hypothesis is that the observed
proportion of a sequence relative to all possible sequences,
\(P^{\zeta}_j\) ( \(\frac{N^\zeta_j}{\sum_j N^\zeta }\)), is equal to
the expected proportion of the sequence relative to all possible
sequences, \(\hat{P}^\zeta_j\). The test statistic is

\begin{equation}\label{eq:prop}
z^\zeta_j = \frac{P^\zeta_j - \hat{P}^\zeta_j}{\sqrt{P^\zeta_{j,\delta} (1 - P^\zeta_{j,\delta}) (1 / N^\zeta_j + 1 / \hat{N}^\zeta_j)  } }
\end{equation}

where \(N^\zeta_j\) and \(\hat{N}^\zeta_j\) respectively represent the
observed and expected number of occurrences of sequence \(\zeta_j\), and
where
\(P^\zeta_{j,\delta} = (P^\zeta_j - \hat{P}^\zeta_j) / ({N^\zeta_j - \hat{N}^\zeta_j})\).
The null hypothesis is that the test statistic is normally distributed.
If we can reject the null hypothesis for such sequences, we can build an
argument in support of the hot hand effect.\footnote{As with the
  binomial tests for the notional and MS post-streak proportions, we
  adjust the \(p\)-values with the BY correction.}

\hypertarget{results}{%
\section{Results}\label{results}}

\hypertarget{distribution-of-rounds-played-1}{%
\subsection{Distribution of rounds
played}\label{distribution-of-rounds-played-1}}

Using Equation \ref{eq:chi-squ}, we find that \(\chi^2 = 16.0\)
(\(p\)-value of \textless0.01) for \(\phi_0 = 0.5\). Thus, we can
comfortably reject the constant probability hypothesis for the naive
\(\phi_0 = 0.5\).

Table \ref{tbl:prob-series-lasts-r-rounds} shows the expected proportion
of series ending in \(r\) rounds given \(\phi_0 = 0.5\),
\(\hat{\Phi}_0(r)\), as well as the observed proportions, \(\Phi(r)\).

\begin{table}
\caption{The observed frequencies of CoD SnD series ending in $r$ rounds shown as a count $N(r)$ and as a proportion $\Phi(r)$ of all series ($\sum_{r \in R} N(r)$, where $r \in R = [6, 7, 8, 9, 10, 11]$). Additionally, the expected proportion, $\hat{\Phi}_0(r)$, under the assumption that each team has a constant round win probability $\phi_0 = 0.5$.}

\centering
\begin{tabular}{rrrr}
\toprule
$r$ & $N(r)$ & $\Phi(r)$ & $\hat{\Phi}_0(r)$ \\ 
\midrule

6 & 40 & $4.7\%$ & $3.1\%$ \\ 
7 & 101 & $11.9\%$ & $9.4\%$ \\ 
8 & 141 & $16.5\%$ & $16.4\%$ \\ 
9 & 185 & $21.7\%$ & $21.9\%$ \\ 
10 & 183 & $21.5\%$ & $24.6\%$ \\ 
11 & 202 & $23.7\%$ & $24.6\%$ \\ 

\bottomrule
\end{tabular}

\label{tbl:prob-series-lasts-r-rounds}

\end{table}

Table \ref{tbl:mosteller-methods-results} shows the alternate values for
the constant round win probability assumption, each approximately equal
to 0.575. When applying Equation \ref{eq:chi-squ}, each results in a
\(\chi^2\) value for which we cannot reject the constant probability
null hypothesis.

\begin{table}

\caption{Alternate estimates of the constant round win probability ($\phi$) and their corresponding $\chi^2$ values, applying the three methods suggested by Mosteller to CoD SnD. Additionally, the naive constant round win probability $\phi_0$ and its $\chi^2$.}

\centering
\begin{tabular}{lrr}
\toprule
Method & $\phi$ & $\chi^2$ ($p$-value) \\
\midrule

$\phi_0$: Naive & 0.5000 & 16.0 ($\leq$ 0.01) \\
$\phi_1$: Method of moments & 0.5725 & 3.6 (0.6) \\
$\phi_2$: Maximum likelihood & 0.5750 & 3.5 (0.62) \\
$\phi_3$: Minimum $\chi^2$ & 0.5775 & 3.5 (0.62) \\

\bottomrule
\end{tabular}

\label{tbl:mosteller-methods-results}

\end{table}

Table \ref{tbl:alternative-constant-ps} shows the expected proportions
of series ending in \(r\) rounds, \(\hat{\Phi}_2(r)\), when re-applying
Equation \ref{eq:series-length} for the maximum likelihood estimate
\(\phi_2 = 0.575\).\footnote{The method of moments and minimum
  \(\chi^2\) estimates for \(\phi\) are omitted simply because the
  results would be nearly identical to those for the maximum likelihood
  estimate of \(\phi\) (since they are all \(\approx 0.575\)).} We
observe that \(\hat{\Phi}_2(r)\) more closely matches \(\Phi(r)\) than
\(\hat{\Phi}_0(r)\).

\begin{table}

\caption{The observed proportion of CoD SnD series ending in $r$ rounds, $\Phi(r)$, compared to the expected proportion, $\hat{\Phi}_2(r)$, under the maximum likelihood estimate, $\phi_2 = 0.575$. Additionally, the expected proportion, $\hat{\Phi}_0(r)$, under the naive assumption $\phi_0 = 0.5$ is repeated from Table \ref{tbl:prob-series-lasts-r-rounds} for comparison.}

\centering
\begin{tabular}{rrrr}
\toprule
$r$ & $\Phi(r)$ & $\hat{\Phi}_0(r)$ = 0.5 & $\hat{\Phi}_2(r)$ = 0.575 \\
\midrule

6 & 4.7\% & 3.1\% & 4.2\% \\
7 & 11.9\% & 9.4\% & 11.2\% \\
8 & 16.5\% & 16.4\% & 17.8\% \\
9 & 21.7\% & 21.9\% & 21.8\% \\
10 & 21.5\% & 24.6\% & 23.0\% \\
11 & 23.7\% & 24.6\% & 22.0\% \\

\bottomrule
\end{tabular}

\label{tbl:alternative-constant-ps}

\end{table}

The observation that \(\hat{\Phi}_2(r)\) reasonably matches \(\Phi(r)\)
supports the implication from the failure to reject the null hypothesis
for \(\phi_2\) (shown in Table \ref{tbl:mosteller-methods-results}): the
constant round win probability assumption can be valid for CoD SnD with
the appropriate choice of \(\phi\) (\(\approx 0.575\)).

\hypertarget{momentum-1}{%
\subsection{Momentum}\label{momentum-1}}

\hypertarget{post-streak-probability}{%
\subsubsection{Post-streak probability}\label{post-streak-probability}}

Given that people typically perceive streaks as beginning after the
third success (or failure) at minimum
(\protect\hyperlink{ref-carlson2007}{Carlson and Shu 2007}), we focus on
streaks of three round wins. Table \ref{tbl:pwkr} compares the observed
round win proportion, \(P^{+|k=3,r}\), with the notional proportion,
\(\hat{P}^{+|k=3,r}\), following streaks of \(k=3\) round wins. The MS
proportion, \(\hat{P}^{+|k=3,r}_{MS}\), is also shown as a reference.

\begin{table}

\caption{For streaks of $k=3$ round wins in CoD SnD series ending in $r$ rounds, the observed count of rounds wins, $r^{+|k=3,r}$, and proportion of round wins, $P^{+|k=3,r}$, among $N^{k=3,r}$ chances ($P^{+|k=3,r} = r^{+|k=3,r} / N^{k=3,r}$). Additionally, the notional and MS expected proportions, $\hat{P}^{+|k=3,r}_0$ and $\hat{P}^{+|k=3,r}_{MS}$, respectively.
}

\centering
\begin{tabular}{rcrrrrr}
\toprule
$r$ & \text{Win series?} & $r^{+|k=3,r}$ & $N^{k=3,r}$ & $P^{+|k=3,r}$ & $\hat{P}^{+|k=3,r}_0$ & $\hat{P}^{+|=k=3,r}_{MS}$ \\ 
\midrule

7 & yes & 156 & 209 & 74.6\% & 75.0\% & 75.7\% \\ 
8 & yes & 130 & 209 & 62.2\% & 60.0\% & 61.9\% \\ 
9 & yes & 100 & 193 & 51.8\% & 50.0\% & 52.7\% \\ 
10 & no & 8 & 60 & 13.3\% & 14.3\% & 26.7\% \\ 
10 & yes & 66 & 151 & 43.7\% & 42.9\% & 44.7\% \\ 
11 & no & 31 & 129 & 24.0\% & 25.0\% & 30.8\% \\ 
11 & yes & 60 & 150 & 40.0\% & 37.5\% & 38.8\% \\ 

\bottomrule
\end{tabular}

\label{tbl:pwkr}

\end{table}

All binomial test \(p\)-values for both \(\hat{P}^{+|k=3,r}_0\) and
\(\hat{P}^{+|k=3,r}_{MS}\) are found to be insignificant at the
\(\alpha = 0.05\) threshold, implying that we cannot reject the null
hypothesis that the notional and MS post-streak round win proportions
are different than the observed proportion for streaks of
three.\footnote{\(p\)-values are not shown for readability purposes.}
When performing the same tests for streaks of two, four, and five, there
is no case in which we can reject the null hypothesis for
\(\hat{P}^{+|k,r}_0\). The null hypothesis can only be rejected in some
extreme cases for \(\hat{P}^{+|k,r}_{MS}\).\footnote{These cases are:
  (1) \(k = 4\), \(r = 7\) (\(N^{k,r} = 108\)); (2) \(k = 4\),
  \(r = 10\) (\(N^{k,r} = 45\)); (3) \(k = 5, r = 7\)
  (\(N^{k,r} = 37\)); (4) \(k = 5, r = 8\) (\(N^{k,r} = 4\)). In each
  case, the team won the series.}

Table \ref{tbl:pw3ri} shows the observed and notional proportions of
round wins---\(P^{+|k=3,r,i}\) and \(\hat{P}^{+|k=3,r,i}_0\)
respectively---immediately following streaks of three round wins when
explicitly accounting for the post-streak round index,
\(i\).\footnote{We omit the MS proportions \(\hat{P}^{+|k=3,r,i}_{MS}\)
  since sample sizes are small and results may or may not be valid due
  to the questionable i.i.d assumption.}\footnote{Some combinations of
  \(r\) and \(i\) are not shown in the table for brevity.} While some of
the observed and notional proportions differ strongly, the BY-corrected
\(p\)-values indicate that the null hypothesis cannot be rejected for
any combination of \(i\) and \(r\) (including those not
shown).\footnote{There are combinations for which we could reject the
  null hypothesis if we did apply the BY correction, such as for
  \(r = 10\) and \(i = 9\).} The same is true for streaks of two, four,
and five.

\begin{table}

\caption{For streaks of $k=3$ round wins in CoD SnD series ending in $r$ rounds, the observed count of rounds wins, $r^{+|k=3,r,i}$, and proportion of round wins, $P^{+|k=3,r,i}$, among $N^{k=3,r,i}$ chances, given the index of the round, $i$, immediately following the streak. Additionally, the notional proportion, $\hat{P}^{+|k=3,r,i}_0$. Table restricted to $r \in [10, 11]$ and $r - 3 \leq i \leq r - 1$.}

\centering
\begin{tabular}{rrcrrrr}
\toprule
$r$ & $i$ & \text{Win series?} & $r^{+|k=3,r,i}$ & $N^{k=3,r,i}$ & $P^{+|k=3,r,i}$ & $\hat{P}^{+|k=3,r,i}_0$\\ 
\midrule

10 & 7 & no & 2 & 6 & 33.3\% & 33.3\% \\ 
10 & 7 & yes & 11 & 27 & 40.7\% & 33.3\% \\ 
10 & 8 & no & 1 & 13 & 7.7\% & 33.3\% \\ 
10 & 8 & yes & 9 & 24 & 37.5\% & 33.3\% \\ 
10 & 9 & no & 2 & 12 & 16.7\% & 50.0\% \\ 
10 & 9 & yes & 4 & 20 & 20.0\% & 50.0\% \\ 
11 & 8 & no & 2 & 17 & 11.8\% & 28.6\% \\ 
11 & 8 & yes & 3 & 19 & 15.8\% & 28.6\% \\ 
11 & 9 & no & 5 & 12 & 41.7\% & 28.6\% \\ 
11 & 9 & yes & 10 & 20 & 50.0\% & 28.6\% \\ 
11 & 10 & no & 5 & 14 & 35.7\% & 50.0\% \\ 
11 & 10 & yes & 5 & 22 & 22.7\% & 50.0\% \\

\bottomrule
\end{tabular}

\label{tbl:pw3ri}

\end{table}

\hypertarget{wald-wolfowitz-runs-test-1}{%
\subsubsection{Wald-Wolfowitz runs
test}\label{wald-wolfowitz-runs-test-1}}

In Table \ref{tbl:ww-sequences}, the observed and expected count of
sequences respectively, are shown for sequence for which we can reject
the Wald-Wolfowitz null hypothesis\footnote{Interestingly, we identify
  the same non-random sequences when using either \(\phi = 0.575\) or
  \(\phi = 0.5\).} at a confidence level of \(\alpha = 0.05\), for
\(r \in [6, 7, 8, 9, 10]\)\footnote{Since the normal method is invalid
  for sequences with only successes or failures, we use the exact method
  to evaluate (and reject) the null hypothesis for the Wald Wolfowitz
  runs test for \(r = 6\).}. In addition to the 14 sequences shown,
there are 12 additional sequences for \(r = 11\).\footnote{Non-random
  sequences for \(r = 11\) excluded for brevity.}

\begin{table}

\caption{Sequences ($\zeta_j$) for which we can reject Wald-Wolfowitz null hypothesis for $r \in [6, 7, 8, 9, 10]$, where round wins and losses are denoted with $+$ and $-$ respectively. Additionally, the observed count, $N^{\zeta}_j$, and proportion, $P^{\zeta}_j$, of all possible sequences, as well as the expected count, $\hat{N}^{\zeta}_j$, and proportion, $\hat{P}^{\zeta}_j$. $\hat{N}^{\zeta}_j$ is scaled to the observed number of series played, hence its non-integer value.}

\centering
\begin{tabular}{rlrrrr}
\toprule
$r$ & $\zeta_j$ & $N^{\zeta}_j$ & $P^{\zeta}_j$ & $\hat{N}^{\zeta}_j$ & $\hat{P}^{\zeta}_j$ \\ 
\midrule
6 & + + + + + + & 40 & 4.69\% & 32.9 & 3.86\% \\
8 & - - + + + + + + & 4 & 0.47\% & 7.67 & 0.90\% \\ 
9 & - - - + + + + + + & 4 & 0.47\% & 3.75 & 0.44\% \\ 
10 & + + + + - - - - + + & 1 & 0.12\% & 2.39 & 0.28\% \\ 
10 & + - - - - + + + + + & 0 & 0.00\% & 2.04 & 0.24\% \\ 
10 & + - + - + + - + - + & 3 & 0.35\% & 2.04 & 0.24\% \\ 
10 & + + + - - - - + + + & 2 & 0.23\% & 2.04 & 0.24\% \\ 
10 & - - - - + + + + + + & 2 & 0.23\% & 1.79 & 0.21\% \\ 
10 & + - + - + - + + - + & 3 & 0.35\% & 1.70 & 0.20\% \\ 
10 & + - + + - + - + - + & 0 & 0.00\% & 1.70 & 0.20\% \\ 
10 & + + - - - - + + + + & 1 & 0.12\% & 1.19 & 0.14\% \\ 
10 & + - + - + - + - + + & 3 & 0.35\% & 1.11 & 0.13\% \\ 
10 & + + - + - + - + - + & 0 & 0.00\% & 1.11 & 0.13\% \\ 
10 & + + + + + - - - - + & 2 & 0.23\% & 1.02 & 0.12\% \\ 

\bottomrule
\end{tabular}

\label{tbl:ww-sequences}
\end{table}

The expected proportion of occurrences of each sequence \(\zeta_j\) is
based on 10,000 simulations using the constant round probability
\(\phi_2 = 0.575\). A test for the difference between the observed and
expected proportions of all sequences, \(P^{\zeta}_j\) and
\(\hat{P}^{\zeta}_j\) respectively, indicates that the null
hypothesis---that the two proportions are equal---cannot be rejected for
any of the non-random sequences.

\hypertarget{discussion}{%
\section{Discussion}\label{discussion}}

Anecdotally, most observers swear by the existence of momentum in CoD
SnD series, to the extent that vernacular has been developed to describe
such phenomenon. Viewers have come to embrace the ``5-3'' phenomenon,
where teams win three consecutive rounds after facing a 5-3 deficit to
win 6-5. There is even a term for the rare 0-5 comeback---a ``full
sail''.

However, our results do not provide evidence for momentum on several
fronts: (1) the evidence supporting the constant round probability
hypothesis; (2) the failure to reject the binomial null hypothesis for
post-streak win probabilities; and (3) the failure to reject the equal
proportions null hypothesis for non-random sequences identified by the
Wald-Wolfowitz runs test.

Perhaps it is not surprising that we did not find evidence in favor of
the hot hand effect given the small ``skill gap'' in the CoD relative to
other esports. (Most professional esports players who have played CoD,
including CoD players themselves, would not hesitate to state this.) A
small skill gap fosters randomness in outcomes, implying that a given
team is less likely to go on streaks of consecutive round wins in SnD,
compared to the counterfactual setting where the skill gap is ``large''
and better teams consistently defeat worse teams.

For the sake of brevity, we did not account for offensive or defensive
role when considering win proportions in the rounds immediately
following streaks. It would be interesting to look at whether the round
win percentage is higher after streaks of round wins when the team
starts the streak as the offensive team. Considering streaks of three,
this would mean that the streaking team would be playing defense in the
round immediately following the last streak win. Knowing that teams are
more likely to win on defense---recall \(\bar{\tau} = 47.8%
\)---we should expect that teams on such streaks are more likely to win
the round than teams that go on streaks of three starting with a
defensive round win. On the other hand, streaks of three starting with
an offensive round win are less likely to occur than such streaks
starting with a defensive
win---\(\bar{\tau}^2 (1 - \bar{\tau}) \approx\) 11.9\% in the former
case and \(\bar{\tau} (1 - \bar{\tau})^2 \approx\) 13.0\% in the latter
case---so we would need to properly account for sample size differences.

Another idea would be to account for the quality of the teams, both in
general and on specific maps. Even if we still cannot prove that the hot
hand effect exists when doing so, a null finding would be insightful in
and of itself.

Further, we have contacted the Twitter user ``R11stats'', who privately
tracks in-round player engagements. Data regarding individual
eliminations during each SnD round would facilitate research into
player-specific momentum. While it seems there is no team-level hot hand
effect in CoD SnD, perhaps there is with individual players.

\hypertarget{references}{%
\section*{References}\label{references}}
\addcontentsline{toc}{section}{References}

\hypertarget{refs}{}
\begin{CSLReferences}{1}{0}
\leavevmode\vadjust pre{\hypertarget{ref-ayton2004}{}}%
Ayton, Peter, and Ilan Fischer. 2004. {``The Hot Hand Fallacy and the
Gambler{'}s Fallacy: Two Faces of Subjective Randomness?''} \emph{Memory
\& Cognition} 32 (8): 13691378.

\leavevmode\vadjust pre{\hypertarget{ref-benjamini2001}{}}%
Benjamini, Yoav, and Daniel Yekutieli. 2001. {``The Control of the False
Discovery Rate in Multiple Testing Under Dependency.''} \emph{Annals of
Statistics}, 11651188.

\leavevmode\vadjust pre{\hypertarget{ref-carlson2007}{}}%
Carlson, Kurt A., and Suzanne B. Shu. 2007. {``The Rule of Three: How
the Third Event Signals the Emergence of a Streak.''}
\emph{Organizational Behavior and Human Decision Processes} 104 (1):
113121.

\leavevmode\vadjust pre{\hypertarget{ref-chance2020}{}}%
Chance, Don. 2020. {``Conditional Probability and the Length of a
Championship Series in Baseball, Basketball, and Hockey.''}
\emph{Journal of Sports Analytics} 6 (2): 111--27.
\url{https://doi.org/10.3233/JSA-200422}.

\leavevmode\vadjust pre{\hypertarget{ref-derover2021}{}}%
DeRover, DeMars. 2021. {``Round Win Probabilities Based on Who's Alive +
Time.''}
\url{https://www.reddit.com/r/VALORANT/comments/n3lpoo/round_win_probabilities_based_on_whos_alive_time/}.

\leavevmode\vadjust pre{\hypertarget{ref-gilovich1985}{}}%
Gilovich, Thomas, Robert Vallone, and Amos Tversky. 1985. {``The Hot
Hand in Basketball: On the Misperception of Random Sequences.''}
\emph{Cognitive Psychology} 17 (3): 295314.

\leavevmode\vadjust pre{\hypertarget{ref-miller2018}{}}%
Miller, Joshua B., and Adam Sanjurjo. 2018. {``Surprised by the Hot Hand
Fallacy? A Truth in the Law of Small Numbers.''} \emph{Econometrica} 86
(6): 20192047.

\leavevmode\vadjust pre{\hypertarget{ref-mosteller1952}{}}%
Mosteller, Frederick. 1952. {``The World Series Competition.''}
\emph{Journal of the American Statistical Association} 47 (259): 355380.

\leavevmode\vadjust pre{\hypertarget{ref-steeger2021}{}}%
Steeger, Gregory M., Johnathon L. Dulin, and Gerardo O. Gonzalez. 2021.
{``Winning and Losing Streaks in the National Hockey League: Are Teams
Experiencing Momentum or Are Games a Sequence of Random Events?''}
\emph{Journal of Quantitative Analysis in Sports} 17 (3): 155170.

\leavevmode\vadjust pre{\hypertarget{ref-xenopoulos2022}{}}%
Xenopoulos, Peter, William Robert Freeman, and Claudio Silva. 2022.
{``Analyzing the Differences Between Professional and Amateur Esports
Through Win Probability.''} In, 34183427.

\end{CSLReferences}

\bibliographystyle{unsrt}
\bibliography{references.bib}


\end{document}
