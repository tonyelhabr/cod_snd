% Options for packages loaded elsewhere
\PassOptionsToPackage{unicode}{hyperref}
\PassOptionsToPackage{hyphens}{url}
\PassOptionsToPackage{dvipsnames,svgnames,x11names}{xcolor}
%
\documentclass[
]{article}

\usepackage{amsmath,amssymb}
\usepackage{lmodern}
\usepackage{iftex}
\ifPDFTeX
  \usepackage[T1]{fontenc}
  \usepackage[utf8]{inputenc}
  \usepackage{textcomp} % provide euro and other symbols
\else % if luatex or xetex
  \usepackage{unicode-math}
  \defaultfontfeatures{Scale=MatchLowercase}
  \defaultfontfeatures[\rmfamily]{Ligatures=TeX,Scale=1}
  \setmathfont[]{Latin Modern Math}
\fi
% Use upquote if available, for straight quotes in verbatim environments
\IfFileExists{upquote.sty}{\usepackage{upquote}}{}
\IfFileExists{microtype.sty}{% use microtype if available
  \usepackage[]{microtype}
  \UseMicrotypeSet[protrusion]{basicmath} % disable protrusion for tt fonts
}{}
\makeatletter
\@ifundefined{KOMAClassName}{% if non-KOMA class
  \IfFileExists{parskip.sty}{%
    \usepackage{parskip}
  }{% else
    \setlength{\parindent}{0pt}
    \setlength{\parskip}{6pt plus 2pt minus 1pt}}
}{% if KOMA class
  \KOMAoptions{parskip=half}}
\makeatother
\usepackage{xcolor}
\setlength{\emergencystretch}{3em} % prevent overfull lines
\setcounter{secnumdepth}{5}
% Make \paragraph and \subparagraph free-standing
\ifx\paragraph\undefined\else
  \let\oldparagraph\paragraph
  \renewcommand{\paragraph}[1]{\oldparagraph{#1}\mbox{}}
\fi
\ifx\subparagraph\undefined\else
  \let\oldsubparagraph\subparagraph
  \renewcommand{\subparagraph}[1]{\oldsubparagraph{#1}\mbox{}}
\fi


\providecommand{\tightlist}{%
  \setlength{\itemsep}{0pt}\setlength{\parskip}{0pt}}\usepackage{longtable,booktabs,array}
\usepackage{calc} % for calculating minipage widths
% Correct order of tables after \paragraph or \subparagraph
\usepackage{etoolbox}
\makeatletter
\patchcmd\longtable{\par}{\if@noskipsec\mbox{}\fi\par}{}{}
\makeatother
% Allow footnotes in longtable head/foot
\IfFileExists{footnotehyper.sty}{\usepackage{footnotehyper}}{\usepackage{footnote}}
\makesavenoteenv{longtable}
\usepackage{graphicx}
\makeatletter
\def\maxwidth{\ifdim\Gin@nat@width>\linewidth\linewidth\else\Gin@nat@width\fi}
\def\maxheight{\ifdim\Gin@nat@height>\textheight\textheight\else\Gin@nat@height\fi}
\makeatother
% Scale images if necessary, so that they will not overflow the page
% margins by default, and it is still possible to overwrite the defaults
% using explicit options in \includegraphics[width, height, ...]{}
\setkeys{Gin}{width=\maxwidth,height=\maxheight,keepaspectratio}
% Set default figure placement to htbp
\makeatletter
\def\fps@figure{htbp}
\makeatother
\newlength{\cslhangindent}
\setlength{\cslhangindent}{1.5em}
\newlength{\csllabelwidth}
\setlength{\csllabelwidth}{3em}
\newlength{\cslentryspacingunit} % times entry-spacing
\setlength{\cslentryspacingunit}{\parskip}
\newenvironment{CSLReferences}[2] % #1 hanging-ident, #2 entry spacing
 {% don't indent paragraphs
  \setlength{\parindent}{0pt}
  % turn on hanging indent if param 1 is 1
  \ifodd #1
  \let\oldpar\par
  \def\par{\hangindent=\cslhangindent\oldpar}
  \fi
  % set entry spacing
  \setlength{\parskip}{#2\cslentryspacingunit}
 }%
 {}
\usepackage{calc}
\newcommand{\CSLBlock}[1]{#1\hfill\break}
\newcommand{\CSLLeftMargin}[1]{\parbox[t]{\csllabelwidth}{#1}}
\newcommand{\CSLRightInline}[1]{\parbox[t]{\linewidth - \csllabelwidth}{#1}\break}
\newcommand{\CSLIndent}[1]{\hspace{\cslhangindent}#1}

\usepackage{arxiv}
\usepackage{orcidlink}
\usepackage{amsmath}
\usepackage[T1]{fontenc}
\makeatletter
\makeatother
\makeatletter
\makeatother
\makeatletter
\@ifpackageloaded{caption}{}{\usepackage{caption}}
\AtBeginDocument{%
\ifdefined\contentsname
  \renewcommand*\contentsname{Table of contents}
\else
  \newcommand\contentsname{Table of contents}
\fi
\ifdefined\listfigurename
  \renewcommand*\listfigurename{List of Figures}
\else
  \newcommand\listfigurename{List of Figures}
\fi
\ifdefined\listtablename
  \renewcommand*\listtablename{List of Tables}
\else
  \newcommand\listtablename{List of Tables}
\fi
\ifdefined\figurename
  \renewcommand*\figurename{Figure}
\else
  \newcommand\figurename{Figure}
\fi
\ifdefined\tablename
  \renewcommand*\tablename{Table}
\else
  \newcommand\tablename{Table}
\fi
}
\@ifpackageloaded{float}{}{\usepackage{float}}
\floatstyle{ruled}
\@ifundefined{c@chapter}{\newfloat{codelisting}{h}{lop}}{\newfloat{codelisting}{h}{lop}[chapter]}
\floatname{codelisting}{Listing}
\newcommand*\listoflistings{\listof{codelisting}{List of Listings}}
\makeatother
\makeatletter
\@ifpackageloaded{caption}{}{\usepackage{caption}}
\@ifpackageloaded{subcaption}{}{\usepackage{subcaption}}
\makeatother
\makeatletter
\@ifpackageloaded{tcolorbox}{}{\usepackage[many]{tcolorbox}}
\makeatother
\makeatletter
\@ifundefined{shadecolor}{\definecolor{shadecolor}{rgb}{.97, .97, .97}}
\makeatother
\makeatletter
\makeatother
\ifLuaTeX
  \usepackage{selnolig}  % disable illegal ligatures
\fi
\IfFileExists{bookmark.sty}{\usepackage{bookmark}}{\usepackage{hyperref}}
\IfFileExists{xurl.sty}{\usepackage{xurl}}{} % add URL line breaks if available
\urlstyle{same} % disable monospaced font for URLs
\hypersetup{
  pdftitle={They're Clutching up! Team Momentum in Round-Based Esports},
  pdfauthor={Tony ElHabr},
  pdfkeywords={esports},
  colorlinks=true,
  linkcolor={blue},
  filecolor={Maroon},
  citecolor={Blue},
  urlcolor={Blue},
  pdfcreator={LaTeX via pandoc}}

\title{They're Clutching up! Team Momentum in Round-Based Esports}
\author{
Tony ElHabr\\\\Georgia Institute of
Technology\\\\\href{mailto:anthonyelhabr@gmail.com}{anthonyelhabr@gmail.com}}
\date{}
\begin{document}
\maketitle
\begin{abstract}
My research investigates patterns in round win percentages in
professional matches (series) of two popular first-person shooter games,
Call of Duty (Search and Destroy mode) and Valorant.

First, I find evidence in Call of Duty (COD) defying the naive
hypothesis that a series represents a sequence of independent events
(rounds), with each team having a constant 50\% probability of winning a
given round.

Second, I examine post-streak round win probability. For COD, in which
teams play first to 6 round wins (maximum 11 rounds), I find that teams
perform significantly worse than expected after streaks of 2, 3, and 4
wins when series end up going to 9, 10, or 11 rounds, even after
accounting for the ``hot-hand'' phenomenon. In Valorant, which requires
one side to win 13 rounds for the series victory, there is evidence of
performing better than expected after streaks of up to 8 wins, which is
perhaps not shocking given the economy aspect of the game.

Third, I compare win percentages in round one versus all other rounds,
hypothesizing that there may be some advantage on either side when there
is no prior information about how the opponent intends to play a given
map in either game. I find one COD map for which there seems to be a
significant defensive advantage in round one, but none otherwise.

Finally, I evaluate behavior when teams have two rounds left to win the
series, observing a peak in COD offensive win percentages in the 4-4
state, and no such oddity in Valorant.
\end{abstract}
{\bfseries \emph Keywords}
\def\sep{\textbullet\ }

esports

\ifdefined\Shaded\renewenvironment{Shaded}{\begin{tcolorbox}[sharp corners, frame hidden, interior hidden, borderline west={3pt}{0pt}{shadecolor}, enhanced, breakable, boxrule=0pt]}{\end{tcolorbox}}\fi

Sit nunc at convallis fringilla semper! Penatibus vivamus eget malesuada
cursus fames magnis potenti. Venenatis ligula enim conubia laoreet orci
class ligula? Massa sem sed enim risus ut mattis inceptos nisl elementum
mattis. Nostra ultricies habitant donec vitae luctus proin gravida
placerat ac libero imperdiet. Auctor habitasse faucibus ultricies purus
eget sociis ultrices habitasse nec ad aptent lectus rutrum proin potenti
montes sodales posuere convallis tempor erat egestas magna lectus sociis
mollis purus.

Dolor turpis euismod himenaeos interdum felis dictum tempus euismod
tortor aliquam? Eget interdum vehicula laoreet quam. Mollis justo cursus
ad blandit feugiat pulvinar sem sodales bibendum. Leo leo magna pulvinar
metus lacinia nam fringilla maecenas duis dis suscipit aenean natoque
sem metus quam risus sagittis convallis primis sociis id dictumst.

\hypertarget{introduction}{%
\section{Introduction}\label{introduction}}

Amet ultricies dignissim varius potenti suscipit justo phasellus dui
ullamcorper luctus suscipit vulputate. Sagittis enim volutpat ultrices
diam curabitur potenti platea euismod vel libero primis feugiat. Nibh
potenti habitant ultricies dictum sociis class at vehicula eu
vestibulum. Curabitur iaculis.

Consectetur est eleifend turpis morbi magna porttitor bibendum ut
blandit aliquet lacus. Nostra iaculis torquent parturient id fermentum
montes at himenaeos. Magna congue blandit fames egestas at inceptos? Id
maecenas blandit maecenas accumsan ridiculus mollis nullam justo
pulvinar elementum inceptos habitant risus.

\newpage{}

\hypertarget{references}{%
\section*{References}\label{references}}
\addcontentsline{toc}{section}{References}

\hypertarget{refs}{}
\begin{CSLReferences}{0}{0}
\end{CSLReferences}



\end{document}
