\documentclass{article}

\usepackage{arxiv}

\usepackage[utf8]{inputenc} % allow utf-8 input
\usepackage[T1]{fontenc}    % use 8-bit T1 fonts
\usepackage{lmodern}        % https://github.com/rstudio/rticles/issues/343
\usepackage{hyperref}       % hyperlinks
\usepackage{url}            % simple URL typesetting
\usepackage{booktabs}       % professional-quality tables
\usepackage{amsfonts}       % blackboard math symbols
\usepackage{nicefrac}       % compact symbols for 1/2, etc.
\usepackage{microtype}      % microtypography
\usepackage{graphicx}

\title{The Hot Hand Fallacy in Call of Duty Search and Destroy}

\author{
    Tony ElHabr
   \\
     \\
   \\
  \texttt{\href{mailto:anthonyelhabr@gmail.com}{\nolinkurl{anthonyelhabr@gmail.com}}} \\
   \And
    Andrew ElHabr
   \\
    School of Industrial and Systems Engineering (ISyE) \\
    Georgia Institute of Technology \\
  Atlanta, Georgia, USA \\
  \texttt{\href{mailto:andrewelhabr@gatech.edu}{\nolinkurl{andrewelhabr@gatech.edu}}} \\
  }


% tightlist command for lists without linebreak
\providecommand{\tightlist}{%
  \setlength{\itemsep}{0pt}\setlength{\parskip}{0pt}}


% Pandoc citation processing
\newlength{\cslhangindent}
\setlength{\cslhangindent}{1.5em}
\newlength{\csllabelwidth}
\setlength{\csllabelwidth}{3em}
\newlength{\cslentryspacingunit} % times entry-spacing
\setlength{\cslentryspacingunit}{\parskip}
% for Pandoc 2.8 to 2.10.1
\newenvironment{cslreferences}%
  {}%
  {\par}
% For Pandoc 2.11+
\newenvironment{CSLReferences}[2] % #1 hanging-ident, #2 entry spacing
 {% don't indent paragraphs
  \setlength{\parindent}{0pt}
  % turn on hanging indent if param 1 is 1
  \ifodd #1
  \let\oldpar\par
  \def\par{\hangindent=\cslhangindent\oldpar}
  \fi
  % set entry spacing
  \setlength{\parskip}{#2\cslentryspacingunit}
 }%
 {}
\usepackage{calc}
\newcommand{\CSLBlock}[1]{#1\hfill\break}
\newcommand{\CSLLeftMargin}[1]{\parbox[t]{\csllabelwidth}{#1}}
\newcommand{\CSLRightInline}[1]{\parbox[t]{\linewidth - \csllabelwidth}{#1}\break}
\newcommand{\CSLIndent}[1]{\hspace{\cslhangindent}#1}

\usepackage{longtable}
\usepackage{amsmath}
\begin{document}
\maketitle


\begin{abstract}
Our research investigates patterns in round win percentages in
professional Search and Destroy (SnD) matches of the popular
first-person shooter game Call of Duty (CoD). First, we find evidence
suppOrting the hypothesis that round win probability can be modeled as a
constant across rounds in the series, although not at the naive 50\%.
Second, we examine post-streak round win probability, given the series
outcome. Given streak length, series length, and series winner, we find
no evidence that post-streak win rate is significantly different than
expected, suggesting that the perception of momentum at the team level
is deceiving. Wald-Wolfowitz run tests also fail to provide evidence for
the hot hand phenomenon.
\end{abstract}

\keywords{
    esports
   \and
    Call of Duty
   \and
    hot hand
   \and
    runs test
  }

\hypertarget{introduction}{%
\section{Introduction}\label{introduction}}

\hypertarget{call-of-duty-search-and-destroy}{%
\subsection{Call of Duty Search and
Destroy}\label{call-of-duty-search-and-destroy}}

Call of Duty (CoD), first released in 2003, is one of the most popular
first-person shooter (FPS) video game franchises of all-time. The most
popular mode in the competitive scene is ``Search and Destroy'' (SnD),
in which one team tries to destroy one of two designated bomb sites on
the map, defended by the opposing team.\footnote{SnD bears resemblance
  to ``Bomb Defusal'' in Counter-Strike and ``Plant/Defuse'' in
  Valorant, two other FPS games played in more popular professional
  leagues.}

In professional CoD SnD, the two teams\footnote{In the 2020 season,
  teams played with five players each; in the 2021 and 2022 seasons,
  teams played with four players each.} take turns playing offense and
defense every round. The first to six round wins (best-of-11 format) is
declared the series winner.\footnote{Each round has a two-minute time
  limit, not counting the potential extension granted if a bomb is
  planted.} A round can end in one of four ways:

\begin{enumerate}
\def\labelenumi{\arabic{enumi}.}
\tightlist
\item
  One team eliminates all members of the other team prior to a bomb
  plant. (Eliminating team wins.)
\item
  The offensive team eliminates all members of the defensive team after
  a bomb plant.\footnote{\begin{itemize}
    \tightlist
    \item
      The bomb can be picked up by any member of the offensive team.
    \item
      The bomb carrier is not obstructed by carrying the bomb
      (i.e.~movement is the same, weapon usage is the same, etc.).
    \item
      The defense does not get any visual indication for who is carrying
      the bomb.
    \item
      A bomb plant takes five seconds. The timer resets if the player
      stops planting site prior to completing it.
    \item
      A bomb defuse takes seven seconds. The timer resets if the player
      ``drops'' the bomb.
    \item
      Once the bomb is planted, the round timer is reset to 45 seconds.
    \end{itemize}} (Offense wins.)
\item
  The defense defuses the bomb after a bomb plant.\footnote{Often the
    defensive team will try to eliminate all team members prior to
    making the defuse, but in some cases, they may try to ``ninja''
    defuse.} (Defense wins.)
\item
  The offense does not make a plant by the time the round timer ends.
  (Defense wins.)
\end{enumerate}

We adopt the terminology ``series'' to refer to a single best-of-11
matchup, so as to mirror the terminology of playoff series in
professional leagues like the National Basketball Association (NBA),
National Hockey League (NHL), and Major League Baseball (MLB). A
``game'' or a ``match'' in such leagues is analogous to a ``round'' of
CoD SnD for the purposes of this paper.\footnote{Further, these are
  analogous to a trial in statistical analysis of Bernoulli events.}

SnD is not the only game type played in competitive CoD. Teams play in a
head-to-head, best-of-five format, where SnD is always played as the
second and fifth game ``maps'', as most players refer to them (if a
fifth map is needed). The best-of-five matchup could also be called a
``series'', but since we analyze only the SnD game types, we refer to
SnD games as series.

\hypertarget{data}{%
\subsection{Data}\label{data}}

CoD has roughly gone through three eras of professional gaming: (1)
Major League Gaming (MLG) tournaments prior to 2016; (2) the CoD World
League (CWL), initiated in 2016; and (3) the 12-franchise CoD League
(CDL), operating since 2020. The CDL has completed three year-long
``seasons'' as of August 2022.\footnote{CoD is fairly unique compared to
  other esports in that it runs on an annual lifecycle, with releases
  coming in the late fall. A new game under the same brand---Call of
  Duty---is published every year by a rotating set of developers. Each
  new game bears resemblance to past ones, often introducing relatively
  small variations (``improvements'') to graphics, game modes, and other
  facets of gameplay. During the CDL era, the games released have been
  Modern Warfare (2020), Cold War (2021) and Vanguard (2022).}

The data set consists of all SnD matches played in major tournaments and
qualifiers during the CDL era, totaling 7,792 rounds across 852
series.\footnote{288 of the series occur in major tournaments, which are
  considered to be more ``competitive'' than the qualifiers since they
  are played in person (COVID-permitting), whereas the qualifiers are
  played online.} Data has been collected in spreadsheets by community
member ``IOUTurtle''.\footnote{Raw data:
  \url{https://linktr.ee/CDLArchive}. Author:
  \url{https://twitter.com/IOUTurtle}. Processed data can be found at
  \url{https://github.com/tonyelhabr/fps_round_games/blob/master/data/cod_rounds.csv}.}

6-0 sweeps make up 4.7\% of all series, while 6-5 series make up 23.7\%
of series. The observed offensive round win percentage across all rounds
is \(\tau_O\) = 47.8\%.\footnote{Offensive round win percentage has been
  nearly constant across the three games during the CDL era: 1. 47.2\%
  in MW (2020) 2. 47.9\% in Cold War (2021) 3. 48.1\% in Vanguard (2022)}
Table \ref{tbl:o-win-prop-by-series-state} shows round win percentages
by series ``state'' (i.e.~the number of round wins by each team prior to
an upcoming round). Offensive round win rate is not quite constant,
although never veers more than 10\% from \(\tau_O\).

\begin{table}

\caption{Offensive round win rates for the upcoming round, given both the offensive and defensive team's prior number of round wins. Numbers in parentheses are sample sizes.}

\centering
\begin{tabular}{crrrrrr}
\toprule
& \multicolumn{6}{c}{Offense round wins} \\ 
\cmidrule(lr){2-7}
Defense round wins & 0 & 1 & 2 & 3 & 4 & 5 \\ 
\midrule

0 & 47.8\%(852) & 46.6\%(408) & 43.1\%(216) & 43.5\%(115) & 43.3\%(67)  & 40.5\%(37)  \\
1 & 48.6\%(444) & 49.3\%(418) & 51.5\%(309) & 43.4\%(205) & 43.3\%(120) & 39.4\%(99)  \\
2 & 52.8\%(218) & 48.9\%(305) & 48.9\%(315) & 46.6\%(262) & 48.7\%(189) & 42.1\%(133) \\
3 & 54.5\%(123) & 46.0\%(200) & 49.6\%(250) & 45.6\%(248) & 44.4\%(214) & 44.8\%(174) \\
4 & 56.9\%(65)  & 54.5\%(145) & 47.2\%(193) & 44.7\%(228) & 55.2\%(221) & 50.5\%(208) \\
5 & 47.4\%(38)  & 49.4\%(83)  & 47.1\%(136) & 50.9\%(175) & 45.2\%(177) & 46.0\%(202) \\

\bottomrule
\end{tabular}

\label{tbl:o-win-prop-by-series-state}

\end{table}

At tournaments, higher-seeded (``better'') teams choose whether they
want to start on offense in the SnD series played as the second or fifth
map, and their opponent is assigned to start on offense in the SnD map
not chosen.\footnote{Hardpoint has been played as the first and fourth
  maps in a matchup. The variant for the third map was Domination in the
  2020 season, and Control in the 2021 and 2022 seasons.}\footnote{Anecdotally,
  the higher-seeded teams tend to choose to play defense in the first
  round of the SnD series played as the second game, although this
  choice is not consistent across teams, or with the same team over
  time.} Outside of tournaments, teams all play each other twice
throughout the season, constituting qualifiers. Each team gets to play
the higher-seeded role once in their head-to-head matchups with a given
team, despite their win-loss records. Overall, we can argue that
potential bias due to stronger teams playing defense---or, if offense,
if offense happens to be generally advantageous for a given map---is
very minimal.

\hypertarget{literature-review}{%
\section{Literature review}\label{literature-review}}

There have been a handful of studies of the distribution of games played
in a series of a professional sport. Most assume a constant probability
\(\phi\) of a given team winning a game in the series, regardless of the
series state. Mosteller (1952) observed that the American League had
dominated the National League in MLB World Series matchups through 1952,
implying that games should not be modeled with a constant (i.e.~the
null) \(\phi_0 = 0.5\). Mosteller proposed three approaches for
identifying the optimal constant probability value of the stronger team
in the World Series, finding \(\phi \approx 0.65\) in each case: (1)
solving for \(\phi\) from the observed average number of games won by
the loser of the series, which he called the ``method of moments''
approach; (2) maximizing the likelihood that the sample would have been
drawn from a population in which the probability of a team winning a
game is constant across the series (i.e.~maximum likelihood), and (3)
minimizing the chi-squared goodness of fit statistic \(\chi^2\) as a
function of \(\phi\).

Chance (2020) re-examines the constant probability notion in the MLB
World Series (1923--2018), the NBA Finals (1951--2018), and the NHL
Stanley Cup (1939--2018). Chance found strong evidence against the null
hypothesis of \(\phi_0 = 0.5\) in the MLB and NHL championship series
when applying Mosteller's first and second methods.

Momentum, one of most discussed topics in sports analytics, goes
hand-in-hand with a discussion of the nature of series
outcomes.\footnote{We often use ``streaks'' and momentum
  interchangeably, but as Steeger et al.~((2021)) note, momentum implies
  dependence between events, whereas streaking does not.} Two opposing
fallacies are observed in the context of momentum: the ``gambler's
fallacy'' (negative recency) and the ``hot hand fallacy'' (positive
recency). Per Ayton et al. (2004), negative recency is ``the belief
that, for random events, runs of a particular outcome \ldots{} will be
balanced by a tendency for the opposite outcome'', while positive
recency is the expectation of observing future results that match recent
results.

Studying both player streaks and team streaks in basketball, in both
observational and controlled settings. Gilovich et al. (1985),
henceforth GVT, ido not find evidence for the hot hand phenomenon.
However, Miller and Sanjurjo (2018), henceforth MS, provided a framework
for quantifying streak selection bias, which effectively works in the
manner posited by the gambler's fallacy. Specifically, MS say that a
``bias exists in a common measure of the conditional dependence of
present outcomes on streaks of past outcomes in sequential data''
implying that, under i.i.d. conditions, ``the proportion of successes
among the outcomes that immediately follow a streak of consecutive
successes is expected to be strictly less than the underlying
(conditional) probability of success''. When applying streak selection
bias to GVT's data, MS came to the opposite conclusions as GVT.

Other research has borrowed a common approach from the field of quality
control, looking at unlikely sequences of events with the Wald-Wolfowitz
runs test (Peel and Clauset 2015, Steeger et al.~2021). Peel and Clauset
found no evidence for unlikely sequences of scoring events in the NHL,
College Football (CFB), and National Football League (NFL), but did in
the NBA. As a check on their entropy approach to momentum
identification, Steeger et al.~found several NHL teams with sequence of
wins in the 2018-2019 regular season violating the Wald-Wolfowitz null
hypothesis.

\hypertarget{our-contribution}{%
\subsection{Our contribution}\label{our-contribution}}

Despite the plethora of existing research on games played in a series
and momentum in sports, these topics have yet to be investigated
heavily, if at all, in esports. Work has been done to examine
intra-round win probability in other FPS titles such as Counter-Strike
(Xenopoulos, Freeman, and Silva 2022) and Valorant (DeRover 2021), both
of which are round-based like CoD SnD. However, considering
Counter-Strike and Valorant specifically, research on round-level trends
seems non-existent, perhaps for one of several reasons:

\begin{enumerate}
\def\labelenumi{\arabic{enumi}.}
\tightlist
\item
  Both have economic aspects that can create clear advantages for one
  team in a round, given how prior rounds played out. CoD has no such
  equivalent, except for perhaps ``score streaks'', which infrequently
  occur.
\item
  Teams play either offense or defense for many consecutive rounds. (The
  former has a 15-15-1 format and the latter uses a 12-12-1 format.) On
  the other hand, teams in CoD SnD rotate roles every round, analogous
  to a 1-1-1-1-1-1-1 format for home advantage in best-of-seven series
  for professional sports like the MLB, NBA, and NHL.\footnote{1-1-1-1-1-1-1
    is not used today in these leagues, but it was at least once in each
    league.} While theoretically one might be able to account for any
  kind of format, such as a 5-5-1, the rotation of team sides every
  round is convenient for convincing ourselves that rounds could
  reasonably be modeled as i.i.d. Bernoulli trials.
\item
  Both Counter-Strike and Valorant have overtime rules---team must win
  by two rounds---which can make end-of-series sequences difficult to
  model. CoD SnD does not have overtime rules.
\end{enumerate}

To our knowledge, there is no existing public statistical research on
the CoD SnD format, beyond descriptive analysis on social media. While
intra-round trends may be more directly applicable to teams looking for
an advantage on their competition, broader investigation of a concept
like the hot-hand fallacy in a different sport---particularly one that
is less subjected to factors that may be difficult to control for,
e.g.~weather---should be useful as a precedent or supplement for future
researchers.

\hypertarget{methodology}{%
\section{Methodology}\label{methodology}}

\hypertarget{distribution-of-rounds-played}{%
\subsection{Distribution of rounds
played}\label{distribution-of-rounds-played}}

In a best-of-\(s\) format, assuming a constant round win probability
\(\phi\)\footnote{If \(\phi > 0.5\), we can say that this is the
  stronger team.} (where \(0 \leq \phi \leq 1\)) for one
team\footnote{\(\phi\) and other symbols are selected so as to reserve
  symbols like \(p\) for usage in other contexts without causing
  confusion for the reader. Upper-case symbol, e.g.~\(\Phi\), are
  consistently used in this paper to represent proportions, while
  lower-case symbols corresponding to upper-case symbols, e.g.~\(\phi\),
  are used to represent Bernoulli trial success rates. ``Hats'',
  e.g.~\(\hat{\Phi}\), are used to convey expectations, while bare
  symbols symbols convey observational data.}, the expected proportion
of series ending in \(r\) rounds (\(r \leq s\)) is given by Equation
\ref{eq:series-length}.

\begin{equation}\label{eq:m}
m = \frac{s + 1}{2}.
\end{equation}

\begin{equation}\label{eq:series-length}
\hat{\Phi}(r) = \frac{(r - 1)!}{(m - 1)!(s - r)!}(\phi^{m}(1 - \phi)^{r - m} + \phi^{r - m}(1 - \phi)^m).
\end{equation}

For example, assuming \(\phi = 0.5\), the probability of a series ending
in nine rounds in CoD SnD, where \(s=11\), is

\[
\hat{\Phi}(9) = \frac{(9 - 1)!}{(6 - 1)!(11 - 9)!}(0.5^{6}(1 - 0.5)^{9 - 6} + 0.5^{9 - 6}(1 - 0.5)^6) = 56 (0.5^9 + 0.5^9) = 0.21875.
\]

To evaluate the constant round win probability null hypothesis---that
is, that the expected and observed round win rates, \(\hat{\Phi}(r)\)
and \(\Phi(r)\) respectively, are equal to one another---we can compute
the chi-square goodness of fit statistic

\begin{equation}\label{eq:chi-squ}
\chi^2 = \sum_{r \in R} \frac{(\Phi(r) - \hat{\Phi}(r))^2}{\hat{\Phi}(r)},
\end{equation}

in which \(R = [6, 7, 8, 9, 10, 11]\) for CoD SnD.

\hypertarget{momentum}{%
\subsection{Momentum}\label{momentum}}

\hypertarget{ms-post-streak-probability}{%
\subsubsection{MS post-streak
probability}\label{ms-post-streak-probability}}

Let us now consider round win rate immediately following a streak of
\(k\) wins, given that the series lasts \(r\) rounds. Removing our
knowledge of a streak, we might model the Bernoulli round win
probability as

\begin{equation}\label{eq:pwr}
p_0(\text{win} | r) = p^{+|r}_0 = \begin{cases} 
\frac{m}{r} & \text{team wins series}, \\ 
\frac{r - m}{r} & \text{team loses series},
\end{cases}
\end{equation}

using \(m\) from Equation \ref{eq:m}.

As shown by MS with their Theorem 1, we should expect the proportion of
round wins immediately following a streak of \(k\) rounds wins for a
series lasting \(r\) rounds, \(\hat{P}^{+|k,r}_{MS}\), to be strictly
less than \(p^{+|r}\).\footnote{See Appendix E.1 of MS for the proof.}
Unfortunately, there does not exist a closed form representation for the
expected value of \(\hat{P}^{+|k,r}_{MS}\) for \(k > 1\). Nonetheless,
one may run simulations to estimate the expected value, as we choose to
do.\footnote{We have adapted the code from Vafa (2017), which implements
  MS's framework (2018).}

Although one might be tempted to mirror the hypothesis testing for
difference of proportions performed by MS (and GVT), our context is
fundamentally different from that of MS, who focus on longitudinal data
in controlled settings.\footnote{GVT also perform statistical tests on
  shots from players in live games, i.e.~``observational'' data, but
  they note that their findings are likely affected player shot
  selection in the face of defensive strategy by the opposing team.} The
number of trials is fixed in their experimental designs, but in CoD SnD,
the number of rounds played is determined as a function of the max
number of possible rounds (\(s\)) and whether or not the team wins the
series. The Bernoulli trial success probability, i.e.~the single round
win rate in CoD SnD, is not independent of the opponent. Consequently, a
statistical test of the difference in \(\hat{P}^{+|k,r}\) and
\(\hat{P}^{-|kr}\), as performed by MS to evaluate their hypothesis
regarding post-streak success rate, is not completely appropriate,
although useful as a reference.

\hypertarget{notional-post-streak-probability}{%
\subsubsection{``Notional'' post-streak
probability}\label{notional-post-streak-probability}}

One can consider another form of the expected proportion of rounds won
immediately after a streak of \(k\) round wins in a best-of-\(s\) series
given the length of the series (\(r\) rounds), the ``notional''
proportion \(\hat{P}^{+|k,r}_0\). The Bernoulli round win probability
underlying \(\hat{P}^{+|k,r}_0\) is

\begin{equation}\label{eq:pwkr}
p_0(\text{win} | k, r) = p^{+|k,r}_0 = \begin{cases}
\frac{m - k}{r - k} & \text{team wins series}, \\
\frac{s - m - k}{r - k} & \text{team loses series},
\end{cases}
\end{equation}

using \(m\) from Equation \ref{eq:m}.

We can perform a binomial test to evaluate the null hypothesis
\(\omega = \omega_0\) for the observed probability of success \(\omega\)
and a user-specified \(\omega_0\) (where \(0 \leq \omega_0 \leq 1\)). If
there are \(r^+\) observed successes in a sample of \(r\) trials and we
expect that there should be \(r * \omega_0\), the probability of
arriving at this expected number of successes is

\begin{equation}\label{eq:binom}
\Pr(r^w) = {\binom {r}{r^+}} \omega^{r^+}(1-\omega)^{r-r^+}.
\end{equation}

Treating the notional proportion \(\hat{P}^{+|k,r}_0\) as the null
\(\omega_0\) and plugging in the observed proportion \(P^{+|k,r}\) for
\(\omega\) in Equation \ref{eq:binom} (treating proportions as
probabilities), we can evaluate the null hypothesis that the observed
probability is equal to the notional probability. If we can reject this
null hypothesis, then we can consider team momentum, represented by
post-streak success, plausible.

One can perform the same binomial test for the MS's
streak-selection-adjusted proportion,
\(\hat{P}^{+|k,r}_{MS}\).\footnote{Given the caveats mentioned before,
  these results should be heeded with caution.}

We can further decompose Equation \ref{eq:pwkr} by the round \(i\)
(where \(i \leq r\)) in which the streak of length \(k\) carries into.

\begin{equation}\label{eq:pwkri}
p_0(\text{win} | k, r, i) = p^{+|k,r,i}_0 = \begin{cases}
\frac{m - k}{r - k}, & \text{team wins series}, i \neq r, \\
1, & \text{team wins series}, i = r, \\
\frac{s - m - k}{r - k}, & \text{team loses series}, i \neq r, \\
0, & \text{team loses series}, i = r.
\end{cases}
\end{equation}

Again, we caapply a binomial test to evaluate the hypothesis that the
expected proportions, \(\hat{P}^{+|k,r,i}_0\) and
\(\hat{P}^{+|k,r,i}_{MS}\) separately, are equal to the observed
proportion \(P^{+|k,r}\).

\hypertarget{wald-wolfowitz-runs-test}{%
\subsubsection{Wald-Wolfowitz runs
test}\label{wald-wolfowitz-runs-test}}

Stepping back from the one-sided perspective of a single team's round
win probability when streaking, one can attempt to detect the hot hand
phenomenon with a Wald-Wolfowitz runs test. Under the null hypothesis,
the number of runs in a sequence of \(r\) trials, \(\zeta(r)\), is a
random variable that can take on values \(r^+\) and \(r^- = r - r^+\)
for the number of success and failure respectively, with the following
mean \(\mu\) and variance \(\sigma^2\):

\begin{equation}\label{eq:ww}
\mu = \frac{2r^{+}r^{-}}{r} + 1, \sigma^2 = \frac{(\mu-1)(\mu-2)}{r-1}.
\end{equation}

To incorporate our findings regarding constant round win probability, We
can specify that \(\Pr(r^+) = \phi\) (and, conversely, that
\(\Pr(r^-) = 1 - \phi\)).

One can subset the observed series sequences to those that violate the
null hypothesis for the run test and perform a test of equal
proportions, where the null is that the observed proportion of a
sequence relative to all possible sequences, \(P(\zeta) = P^\zeta\), is
equal to the expected proportion of the sequence relative to all
possible sequences, \(\hat{P}^\zeta\). The test statistic is

\begin{equation}\label{eq:prop}
Z = \frac{P^\zeta - \hat{P}^\zeta}{\sqrt{P^\zeta_{\delta} (1 - P^\zeta_{\delta}) (1 / N^\zeta + 1 / \hat{N}^\zeta)  } }
\end{equation}

where

\[
P^\zeta_{\delta} = \frac{P^\zeta - \hat{P}^\zeta}{N^\zeta - \hat{N}^\zeta}
\]

and where \(N^\zeta\) is the observed number of sequences,
\(\hat{N}^\zeta\) is the expected number of sequences. If we can reject
the null hypothesis for such sequences, then there is implicit evidence
in favor of the hot hand.

\hypertarget{results}{%
\section{Results}\label{results}}

First, we investigate the constant probability assumption and the
distribution of rounds played in a series. Chance's work is closely
related to our, and, in fact, provides a guide for this investigation.
Afterwards, We investigate post-streak win rates, referring to work from
MS. We finish with a brief look at sequences of round wins violating the
Wald-Wolfowitz runs test, and whether such observed sequences occur more
frequently than expected.

\hypertarget{distribution-of-rounds-played-1}{%
\subsection{Distribution of rounds
played}\label{distribution-of-rounds-played-1}}

Using Equation \ref{eq:chi-squ}, we find that \(\chi^2 = 16.0\)
(\(p\)-value of 0.0068) for \(\phi_0 = 0.5\). Thus, we can comfortably
reject the constant probability hypothesis for the null
\(\phi_0 = 0.5\), even at a confidence level of \(\alpha = 0.01\).

Table \ref{tbl:prob-series-lasts-r-rounds} shows the expected series
lasting \(r\) rounds, the expected proportion of series given
\(\phi_0 = 0.5\), \(\hat{\Phi}_0(r)\), and the observed proportions,
\(\Phi(r)\).

\begin{table}
\caption{The expected proportion of CoD SnD series lasting $r$ rounds, $\hat{\Phi}_0(r)$, under the assumption that each team has a constant round win probability $\phi_0 = 0.5$. Additionally, the observed frequencies for CoD SnD shown as a count $N(r)$ and as a proportion $\Phi(r)$ of all series ($\sum_{r \in R} N(r)$, where $r \in R = [6, 7, 8, 9, 10, 11]$).}

\centering
\begin{tabular}{rrrr}
\toprule
$r$ & $N(r)$ & $\Phi(r)$ & $\hat{\Phi}_0(r)$ \\ 
\midrule

6 & 40 & $4.7\%$ & $3.1\%$ \\ 
7 & 101 & $11.9\%$ & $9.4\%$ \\ 
8 & 141 & $16.5\%$ & $16.4\%$ \\ 
9 & 185 & $21.7\%$ & $21.9\%$ \\ 
10 & 183 & $21.5\%$ & $24.6\%$ \\ 
11 & 202 & $23.7\%$ & $24.6\%$ \\ 

\bottomrule
\end{tabular}

\label{tbl:prob-series-lasts-r-rounds}

\end{table}

Table \ref{tbl:mosteller-methods-results} shows the alternate values for
the constant round win probability that we find when applying the three
methods suggested by Mosteller (1952). Each is approximately equal to
0.575. When applying Equation \ref{eq:chi-squ}, each results in a
\(\chi^2\) value for which we cannot reject the constant probability
null hypothesis.

\begin{table}

\caption{Alternate estimates of the constant probability ($\phi$) for winning a given round in a CoD SnD, applying the three methods suggested by Mosteller (1952), in addition to the naive ($\phi_0 = 0.5$).}

\centering
\begin{tabular}{lrr}
\toprule
Method & $\phi$ & $\chi^2$ ($p$-value) \\
\midrule

0. Naive & 0.5000 & 16.0 ($\leq$ 0.01) \\
1. Method of moments & 0.5725 & 3.6 (0.6) \\
2. Maximum likelihood & 0.5750 & 3.5 (0.62) \\
3. Minimum ($\chi^2$) & 0.5775 & 3.5 (0.62) \\

\bottomrule
\end{tabular}

\label{tbl:mosteller-methods-results}

\end{table}

Table \ref{tbl:alternative-constant-ps} shows the new \(\hat{\Phi}(r)\)
when re-applying Equation \ref{eq:series-length} for the maximum
likelihood estimate \(\phi_2 = 0.575\), resulting in a new set of
expected proportions of series lasting \(r\) rounds
\(\hat{\Phi}_2(r)\).\footnote{The method of moments and minimum
  \(\chi^2\) estimates for \(\phi\) are omitted simply because the
  results would be nearly identical to those for the maximum likelihood
  estimate of \(\phi\) (since they are all \(\approx 0.575\)).} We
observe that \(\hat{\Phi}_2(r)\) is larger than \(\hat{\Phi}_0(r)\) for
\(r \in [6, 7]\), more closely matching \(\Phi(r)\). \(\hat{\Phi}_2(r)\)
is also closer to the observed \(\Phi(r)\) for \(r \in [9, 10]\),
although not for \(r \in [8, 11]\).

\begin{table}
\caption{The observed proportion of CoD SnD series, $\Phi(r)$, ending in $r$ rounds, compared to the expected proportion, $\hat{\Phi}_0(r)$, under the naive assumption $\phi_0 = 0.5$ and the expected proportion, $\hat{\Phi}_2(r)$, under the maximum likelihood estimate, $\phi_2 = 0.575$, for constant round win probability.}

\centering
\begin{tabular}{rrrrrr}
\toprule
$r$ & $\Phi(r)$ & $\hat{\Phi}_0(r)$ = 0.5 & $\hat{\Phi}_2(r)$ = 0.575 \\
\midrule

6 & 4.7\% & 3.1\% & 4.2\% \\
7 & 11.9\% & 9.4\% & 11.2\% \\
8 & 16.5\% & 16.4\% & 17.8\% \\
9 & 21.7\% & 21.9\% & 21.8\% \\
10 & 21.5\% & 24.6\% & 23.0\% \\
11 & 23.7\% & 24.6\% & 22.0\% \\

\bottomrule
\end{tabular}

\label{tbl:alternative-constant-ps}

\end{table}

Observing that \(\hat{\Phi}_2(r)\) reasonably matches \(\Phi(r)\)
(especially in comparison to \(\hat{\Phi}_0(r)\)), along with the null
hypothesis rejection shown in Table \ref{tbl:mosteller-methods-results},
we can say that the constant round win probability assumption is valid
in CoD SnD series with the appropriate choice of \(\phi\)
(\(\approx 0.575\)).

\hypertarget{momentum-1}{%
\subsection{Momentum}\label{momentum-1}}

\hypertarget{post-streak-probability}{%
\subsubsection{Post-streak probability}\label{post-streak-probability}}

Given that people typically perceive streaks as beginning after the
third success (or failure) at minimum (Carlson and Shu 2007), we focus
on streaks of three round wins.\footnote{Three happens to also be a
  reasonable number for series that last at maximum 11 rounds.} Table
\ref{tbl:pwkr} compares the notional and MS expected proportions,
\(\hat{P}^{+|k,r}_0\) and \(\hat{P}^{+|k,r}_{MS}\) respectively, with
the observed round win rate, \(P^{+|kr}\), following streaks of \(k=3\)
round wins given the series outcome.

\begin{table}

\caption{Given the round win streak $k=3$, the length of the series ($r$ rounds), and the series winner, the observed count of rounds wins, $r^{+|k=3,r}$, and proportion of round wins, $P^{+|k=3,r}$, among $N^{k=3,r}$ instances where a team could win after the streak ($P^{+|k=3,r} = \frac{r^{+|k=3,r}}{N^{k=3,r}}$). Additionally, the notional and MS expected proportions, $\hat{P}^{+|k=3,r}_0$ and $\hat{P}^{+|k=3,r}_{MS}$ respectively.}

\centering
\begin{tabular}{rcrrrrr}
\toprule
$r$ & \text{Win series?} & $r^{+|k=3,r}$ & $N^{k=3,r}$ & $P^{+|k=3,r}$ & $\hat{P}^{+|k=3,r}_0$ & $\hat{P}^{+|=k=3,r}_{MS}$ \\ 
\midrule

7 & yes & 156 & 209 & 74.6\% & 75.0\% & 75.7\% \\ 
8 & yes & 130 & 209 & 62.2\% & 60.0\% & 61.9\% \\ 
9 & yes & 100 & 193 & 51.8\% & 50.0\% & 52.7\% \\ 
10 & no & 8 & 60 & 13.3\% & 14.3\% & 26.7\% \\ 
10 & yes & 66 & 151 & 43.7\% & 42.9\% & 44.7\% \\ 
11 & no & 31 & 129 & 24.0\% & 25.0\% & 30.8\% \\ 
11 & yes & 60 & 150 & 40.0\% & 37.5\% & 38.8\% \\ 

\bottomrule
\end{tabular}

\label{tbl:pwkr}

\end{table}

With the exception of \(\hat{P}^{+|k,r}_{MS}\) when \(r = 10\) and the
eventual series loser is the team that wins after a streak of three
round wins, all un-adjusted binomial test \(p\)-values are greater than
the \(\alpha = 0.05\) confidence level, implying that we cannot reject
the null hypothesis that the expected notional and MS post-streak round
win rates are different than the observed proportion. However, after
adjusting the \(p\)-values with the Benjamini and Yekutieli (BY)
correction (2001), the null hypothesis cannot be rejected for any
case.\footnote{As noted before, the MS expectations may be unreliable
  for CoD SnD, so one is inclined to prefer the results of the notional
  binomial tests.}

When performing the same tests for streaks of two, four, and five, there
is no case in which we can reject the binomial null hypothesis for the
expected notional proportion \(\hat{P}^{+|k,r}_0\) (even before the BY
p-value correction). The null hypothesis can only be rejected for the
expected MS proportion \(\hat{P}^{+|k,r}_{MS}\) when \(k = 5, r = 7\)
after applying the BY p-value correction.\footnote{At 36, the sample
  size for \(k = 5, r = 7\) is not as large as most other combinations
  of \(k\) and \(r\).}

Now let us explicitly consider the round, \(i\). As shown in Table
\ref{tbl:pw3ri}, the notional win rate in the round \(i\) immediately
following a streak of \(k\) round wins in a series lasting \(r\) rounds,
\(\hat{P}{+|k,r,i}_0\), is statistically different than the observed
\(P^{+|k,r,i}\) in several cases, notably when \(r = i\) for each of
\(r \in [9, 10, 11]\).\footnote{The null can also be rejected for
  \(r \in [7, 8]\), although these are not shown, and the sample sizes
  are smaller.} Interestingly, the observed rate is greater than the
expected rate in each case where we can reject the null hypothesis,
whereas the observed rate is less than the expected rate in all other
cases, as well in a strong majority of cases not shown.

\begin{table}

\caption{Given the round win streak $k=3$, the index of the round immediately following the streak $i$, the length of the series ($r$ rounds), and the series winner, the observed count of rounds wins, $r^{+|k=3,r,i}$, and proportion of round wins, $P^{+|k=3,r,i}$, among $N^{k=3,r,i}$ instances where a team could win after the streak. Additionally, the notional proportion, $\hat{P}^{+|k=3,r,i}_0$. Table restricted to $r \in [10, 11]$, $i < r$, and $N^{k=3,r,i} \geq 10$ for brevity.}

\centering
\begin{tabular}{rrcrrrr}
\toprule
$r$ & $i$ & \text{Win series?} & $r^{+|k=3,r,i}$ & $N^{k=3,r,i}$ & $P^{+|k=3,r,i}$ & $\hat{P}^{+|k=3,r,i}_0$\\ 
\midrule

10 & 7 & no & 2 & 13 & 15.4\% & 14.3\% \\ 
10 & 7 & yes & 11 & 13 & 84.6\% & 42.9\% \\ 
10 & 8 & no & 1 & 10 & 10.0\% & 14.3\% \\ 
10 & 8 & yes & 9 & 10 & 90.0\% & 42.9\% \\ 
11 & 5 & no & 9 & 13 & 69.2\% & 25.0\% \\ 
11 & 5 & yes & 4 & 13 & 30.8\% & 37.5\% \\ 
11 & 6 & no & 4 & 13 & 30.8\% & 25.0\% \\ 
11 & 6 & yes & 9 & 13 & 69.2\% & 37.5\% \\ 
11 & 7 & no & 3 & 10 & 30.0\% & 25.0\% \\ 
11 & 7 & yes & 7 & 10 & 70.0\% & 37.5\% \\ 
11 & 9 & no & 5 & 15 & 33.3\% & 25.0\% \\ 
11 & 9 & yes & 10 & 15 & 66.7\% & 37.5\% \\ 
11 & 10 & no & 5 & 10 & 50.0\% & 25.0\% \\ 
11 & 10 & yes & 5 & 10 & 50.0\% & 37.5\% \\ 

\bottomrule
\end{tabular}

\label{tbl:pw3ri}

\end{table}

\hypertarget{wald-wolfowitz-runs-test-1}{%
\subsubsection{Wald-Wolfowitz runs
test}\label{wald-wolfowitz-runs-test-1}}

In Table \ref{tbl:ww-sequences}, the observed and expected count of
sequences, \(N^{\zeta r}\) and \(\hat{N}^{\zeta r}\) respectively, are
shown for sequences, \(\zeta(r)\), for which one can reject the
Wald-Wolfowitz null hypothesis, given probability of Bernoulli trial
success \(\phi = 0.575\),\footnote{If we used \(\phi = 0.5\), we would
  get the same exact seqeunces.} at a confidence level of
\(\alpha = 0.05\), for \(r \in [8, 9, 10]\). In addition to the 13
sequences shown, there are 12 additional sequences for \(r = 11\).

\begin{table}

\caption{Sequences ($\zeta(r)$) for which we can reject Wald-Wolfowitz null hypothesis for $r \in [8, 9, 10]$, where round wins and losses are denoted with $+$ and $-$ respectively. The observed count, $N^{\zeta r}$, and proportion, $P^{\zeta r}$, of all possible sequences, as well as the expected count, $\hat{N}^{\zeta r}$, and proportion, $\hat{P}^{\zeta r}$. $\hat{N}^{\zeta r}$ is scaled to the observed number of series played, hence its non-integer value.}

\centering
\begin{tabular}{rlrrrr}
\toprule
$r$ & $\zeta(r)$ & $N^{\zeta r}$ & $P^{\zeta r}$ & $\hat{N}^{\zeta r}$ & $\hat{P}^{\zeta r}$ \\ 
\midrule
8 & - - + + + + + + & 4 & 0.47\% & 7.67 & 0.90\% \\ 
9 & - - - + + + + + + & 4 & 0.47\% & 3.75 & 0.44\% \\ 
10 & + + + + - - - - + + & 1 & 0.12\% & 2.39 & 0.28\% \\ 
10 & + - - - - + + + + + & 0 & 0.00\% & 2.04 & 0.24\% \\ 
10 & + - + - + + - + - + & 3 & 0.35\% & 2.04 & 0.24\% \\ 
10 & + + + - - - - + + + & 2 & 0.23\% & 2.04 & 0.24\% \\ 
10 & - - - - + + + + + + & 2 & 0.23\% & 1.79 & 0.21\% \\ 
10 & + - + - + - + + - + & 3 & 0.35\% & 1.70 & 0.20\% \\ 
10 & + - + + - + - + - + & 0 & 0.00\% & 1.70 & 0.20\% \\ 
10 & + + - - - - + + + + & 1 & 0.12\% & 1.19 & 0.14\% \\ 
10 & + - + - + - + - + + & 3 & 0.35\% & 1.11 & 0.13\% \\ 
10 & + + - + - + - + - + & 0 & 0.00\% & 1.11 & 0.13\% \\ 
10 & + + + + + - - - - + & 2 & 0.23\% & 1.02 & 0.12\% \\ 

\bottomrule
\end{tabular}

\label{tbl:ww-sequences}
\end{table}

The relative frequency of the expected proportions are based on 10,000
simulations using the constant round probability \(\phi_2 = 0.575\). A
test for the difference between the observed and expected proportions of
all sequences, \(P^{\zeta r}\) and \(\hat{P}^{\zeta r}\) respectively,
indicates that the null hypothesis---that the two proportions are
equal---cannot be rejected for any of the 25 significant sequences.

\hypertarget{discussion}{%
\section{Discussion}\label{discussion}}

Anecdotally, most observers swear by the existence of momentum in CoD
SnD series, to the extent that vernacular has been developed to describe
such phenomenon. Viewers have come to embrace the ``5-3'' phenomenon,
where teams win three consecutive rounds after facing a 5-3 deficit to
win 6-5. There is even a term for the rare 0-5 comeback---a ``full
sail''.

However, our results do not provide evidence for momentum on several
fronts: (1) the evidence supporting the constant round probability; (2)
the failure to reject the binomial null hypothesis for post-streak win
rates; and (3) the failure to reject the equal proportions null
hypothesis for streaky round sequences identified by the Wald-Wolfowitz
runs test.

Perhaps it is not surprising that we did not find evidence in favor of
the hot hand effect given the small ``skill gap'' in the CoD relative to
other esports. (Most professional esports players who have played CoD,
including CoD players themselves, would not hesitate to state this.) A
small skill gap fosters randomness in outcomes, implying that a given
team is less likely to enjoy streaks of success in SnD.

For the sake of brevity, we did not delve into win rates and streaks
split by offensive or defensive role. For example, it would be
interesting to look at whether the round win rate is higher after a
streaks of three round wins when the team starts the streak as the
offensive team. This would mean that, in the round immediately following
the latest streak win, the streaking team would be playing defense,
where teams are slightly more likely to win on average. On the other
hand, streaks of three starting with an offensive round
win---theoretically, knowing that \(\tau_O\) = 47.7\%, the expected
frequency is \(\tau_O^2 (1 - \tau_O) \approx\) 11.9\%---are likely to
occur less frequently than such streaks starting with a defensive
win---\(\tau_O (1 - \tau_O)^2 \approx\) 13.0\%---so we would want to
account for sample size differences.

In the future, we could account for the quality of the teams, both in
general and on specific maps. Even if doing so does not change the
results, we could gain additional insight into why we do not observe
statistically significant streakiness.

Further regarding future work, we have contacted the Twitter user
``R11stats'', who privately tracks in-round player
engagements.\footnote{\url{https://twitter.com/R11stats}} R11stats
expressed intent on making the data public, which would allow for
research into player-specific momentum. While it seems there is no
team-level hot hand effect in CoD SnD, perhaps there is with
eliminations performed by individual players.

\hypertarget{references}{%
\section*{References}\label{references}}
\addcontentsline{toc}{section}{References}

\hypertarget{refs}{}
\begin{CSLReferences}{1}{0}
\leavevmode\vadjust pre{\hypertarget{ref-ayton2004}{}}%
Ayton, Peter, and Ilan Fischer. 2004. {``The Hot Hand Fallacy and the
Gambler{'}s Fallacy: Two Faces of Subjective Randomness?''} \emph{Memory
\& Cognition} 32 (8): 13691378.

\leavevmode\vadjust pre{\hypertarget{ref-benjamini2001}{}}%
Benjamini, Yoav, and Daniel Yekutieli. 2001. {``The Control of the False
Discovery Rate in Multiple Testing Under Dependency.''} \emph{Annals of
Statistics}, 11651188.

\leavevmode\vadjust pre{\hypertarget{ref-carlson2007}{}}%
Carlson, Kurt A., and Suzanne B. Shu. 2007. {``The Rule of Three: How
the Third Event Signals the Emergence of a Streak.''}
\emph{Organizational Behavior and Human Decision Processes} 104 (1):
113121.

\leavevmode\vadjust pre{\hypertarget{ref-chance2020}{}}%
Chance, Don. 2020. {``Conditional Probability and the Length of a
Championship Series in Baseball, Basketball, and Hockey.''}
\emph{Journal of Sports Analytics} 6 (2): 111--27.
\url{https://doi.org/10.3233/JSA-200422}.

\leavevmode\vadjust pre{\hypertarget{ref-derover2021}{}}%
DeRover, DeMars. 2021. {``Round Win Probabilities Based on Who's Alive +
Time.''}
\url{https://www.reddit.com/r/VALORANT/comments/n3lpoo/round_win_probabilities_based_on_whos_alive_time/}.

\leavevmode\vadjust pre{\hypertarget{ref-gilovich1985}{}}%
Gilovich, Thomas, Robert Vallone, and Amos Tversky. 1985. {``The Hot
Hand in Basketball: On the Misperception of Random Sequences.''}
\emph{Cognitive Psychology} 17 (3): 295314.

\leavevmode\vadjust pre{\hypertarget{ref-miller2018}{}}%
Miller, Joshua B., and Adam Sanjurjo. 2018. {``Surprised by the Hot Hand
Fallacy? A Truth in the Law of Small Numbers.''} \emph{Econometrica} 86
(6): 20192047.

\leavevmode\vadjust pre{\hypertarget{ref-mosteller1952}{}}%
Mosteller, Frederick. 1952. {``The World Series Competition.''}
\emph{Journal of the American Statistical Association} 47 (259): 355380.

\leavevmode\vadjust pre{\hypertarget{ref-steeger2021}{}}%
Steeger, Gregory M., Johnathon L. Dulin, and Gerardo O. Gonzalez. 2021.
{``Winning and Losing Streaks in the National Hockey League: Are Teams
Experiencing Momentum or Are Games a Sequence of Random Events?''}
\emph{Journal of Quantitative Analysis in Sports} 17 (3): 155170.

\leavevmode\vadjust pre{\hypertarget{ref-vafa2017}{}}%
Vafa, Keyon. 2017. {``Is the Hot Hand Fallacy a Fallacy?''}
\url{https://keyonvafa.github.io/hot-hand/}.

\leavevmode\vadjust pre{\hypertarget{ref-xenopoulos2022}{}}%
Xenopoulos, Peter, William Robert Freeman, and Claudio Silva. 2022.
{``Analyzing the Differences Between Professional and Amateur Esports
Through Win Probability.''} In, 34183427.

\end{CSLReferences}

\bibliographystyle{unsrt}
\bibliography{references.bib}


\end{document}
