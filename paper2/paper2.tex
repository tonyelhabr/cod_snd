\documentclass{article}

\usepackage{arxiv}

\usepackage[utf8]{inputenc} % allow utf-8 input
\usepackage[T1]{fontenc}    % use 8-bit T1 fonts
\usepackage{lmodern}        % https://github.com/rstudio/rticles/issues/343
\usepackage{hyperref}       % hyperlinks
\usepackage{url}            % simple URL typesetting
\usepackage{booktabs}       % professional-quality tables
\usepackage{amsfonts}       % blackboard math symbols
\usepackage{nicefrac}       % compact symbols for 1/2, etc.
\usepackage{microtype}      % microtypography
\usepackage{graphicx}

\title{They're Clutching up! Team Momentum in Round-Based Esports}

\author{
    Tony ElHabr
   \\
     \\
   \\
  \texttt{\href{mailto:anthonyelhabr@gmail.com}{\nolinkurl{anthonyelhabr@gmail.com}}} \\
  }


% tightlist command for lists without linebreak
\providecommand{\tightlist}{%
  \setlength{\itemsep}{0pt}\setlength{\parskip}{0pt}}


% Pandoc citation processing
\newlength{\cslhangindent}
\setlength{\cslhangindent}{1.5em}
\newlength{\csllabelwidth}
\setlength{\csllabelwidth}{3em}
\newlength{\cslentryspacingunit} % times entry-spacing
\setlength{\cslentryspacingunit}{\parskip}
% for Pandoc 2.8 to 2.10.1
\newenvironment{cslreferences}%
  {}%
  {\par}
% For Pandoc 2.11+
\newenvironment{CSLReferences}[2] % #1 hanging-ident, #2 entry spacing
 {% don't indent paragraphs
  \setlength{\parindent}{0pt}
  % turn on hanging indent if param 1 is 1
  \ifodd #1
  \let\oldpar\par
  \def\par{\hangindent=\cslhangindent\oldpar}
  \fi
  % set entry spacing
  \setlength{\parskip}{#2\cslentryspacingunit}
 }%
 {}
\usepackage{calc}
\newcommand{\CSLBlock}[1]{#1\hfill\break}
\newcommand{\CSLLeftMargin}[1]{\parbox[t]{\csllabelwidth}{#1}}
\newcommand{\CSLRightInline}[1]{\parbox[t]{\linewidth - \csllabelwidth}{#1}\break}
\newcommand{\CSLIndent}[1]{\hspace{\cslhangindent}#1}

\begin{document}
\maketitle


\begin{abstract}
My research investigates patterns in round win percentages in
professional Search and Destroy (SnD) matches of the popular
first-person shooter game Call of Duty (CoD).

First, I find evidence in CoD defying the naive hypothesis that a series
represents a sequence of independent events (rounds), with each team
having a constant 50\% probability of winning a given round.

Second, I examine post-streak round win probability. I find that teams
perform significantly worse than expected after streaks of 2, 3, and 4
wins when series end up going to 9, 10, or 11 (maximum) rounds, even
after accounting for the ``hot-hand'' phenomenon.
\end{abstract}


\hypertarget{introduction}{%
\section{Introduction}\label{introduction}}

\hypertarget{description-of-call-of-duty-search-and-destroy}{%
\subsection{Description of Call of Duty Search and
Destroy}\label{description-of-call-of-duty-search-and-destroy}}

Call of Duty (CoD), first released in 2003, is one of the most popular
first-person shooter (FPS) video game franchises of all-time. The most
popular mode in the competitive scene is ``Search and Destroy'' (SnD),
which bears resemblance to ``Bomb Defusal'' in Counter-Strike and
``Plant/Defuse'' in Valorant, two other FPS games played in professional
leagues. SnD is one-sided game mode in which one team, the offensive
side, tries to destroy one of two designated bomb sites on the map.

In professional CoD, a team must win six rounds of SnD to win the
match.\footnote{A maximum of 11 even rounds can be played. There is no
  ``sudden death'' or ``win by two'' rule like there are for SnD
  equivalent in professional Counter-Strike and Valorant matches.} A
round can end in one of fmy ways:

\begin{enumerate}
\def\labelenumi{\arabic{enumi}.}
\tightlist
\item
  One team eliminates all members of the other team prior to a bomb
  plant. (Eliminating team wins.)
\item
  The offensive team eliminates all members of the defensive team after
  a bomb plant.\footnote{\begin{itemize}
    \tightlist
    \item
      The bomb can be picked up by any member of the offensive team.
    \item
      The bomb carrier is not obstructed at all by carrying the bomb
      (i.e.~movement is the same, weapon usage is the same).
    \item
      The defense does not get any visual indication for who is carrying
      the bomb.
    \item
      A bomb plant takes five seconds. The timer resets if the player
      stops planting site prior to completing it.
    \item
      A bomb defuse takes seven seconds. The timer resets if the player
      ``drops'' the bomb.
    \item
      The bomb takes 45 seconds to defuse after being planted.
    \end{itemize}} (Offense wins.)
\item
  The defensive team defuses the bomb after a bomb plant.\footnote{Often
    the defensive team will try to eliminate all team members prior to
    making the defuse, but in some cases, they may try to ``ninja''
    defuse.} (Defense wins.)
\item
  The offensive team does not make a plant by the time the round timer
  ends. (Defense wins.)
\end{enumerate}

Teams take turns playing offense and defense every round.

I adopt the terminology ``series'' to refer to what CoD SnD players
typically call a ``match'', so as to emulate the terminology of playoff
series in professional leagues like the National Basketball Association,
National Hockey League, and Major League Baseball. A ``game'' or a
``match'' in such leagues is analogous to a ``round'' of CoD SnD.

\hypertarget{data}{%
\subsection{Data}\label{data}}

CoD has roughly gone through three eras of professional gaming: (1)
Major League Gaming (MLG) tmynaments prior to 2016; (2) the CoD World
League (CWL), initiated in 2016; and (3) the 12-franchise CoD League
(CDL), running since 2020 and completing three year-long ``seasons''
completed as of August 2022.\footnote{CoD is fairly unique compared to
  other esports in that it runs on an annual lifecycle (released coming
  in the late fall), where a new game is published every year under the
  same title. Each new game bears resemblance to past ones, often
  introducing relatively small variations (``improvements'') to
  graphics, game modes, and other facets of gameplay. During the CDL
  era, the games released have been Modern Warfare (2020), Cold War
  (2021) and Vanguard (2022).} The data set consists of all SnD matches
played in tmynaments and qualifiers during the CDL era, totaling 7,792
rounds across 852 series. Data was collected in spreadsheets by
community member ``IOUTurtle''.\footnote{Data:
  \url{https://linktr.ee/CDLArchive}. Author:
  \url{https://twitter.com/IOUTurtle}}

The empirical offensive round win percentage across all rounds is
47.8\%.\footnote{Offensive round win percentage has been nearly constant
  across the three games during the CDL era: 1. 47.2\% in MW (2020) 2.
  47.9\% in Cold War (2021) 3. 48.1\% in Vanguard (2022)} To
contextualize my subsequent discussion of series length and momentum,
table \ref{tbl:cod-o-win-prop-by-series-state} shows round win
percentages by series ``state'' (i.e.~the number of round wins by each
team prior to an upcoming round). Offensive round win rate is not quite
constant, although never veers more than 10\% from this global average.

\begin{longtable}{crrrrrr}
  \caption{Offensive round win rates for the upcoming round, given both the offensive and defensive team's prior number of round wins}\label{tbl:cod-o-win-prop-by-series-state} \\
  \toprule
   & \multicolumn{6}{c}{Offense round wins} \\ 
  \cmidrule(lr){2-7}
  Defense round wins & 0 & 1 & 2 & 3 & 4 & 5 \\ 
    \midrule
    0 & 47.8\%
    (852) & 46.6\%
    (408) & 43.1\%
    (216) & 43.5\%
    (115) & 43.3\%
    (67) & 40.5\%
    (37) \\ 
    1 & 48.6\%
    (444) & 49.3\%
    (418) & 51.5\%
    (309) & 43.4\%
    (205) & 43.3\%
    (120) & 39.4\%
    (99) \\ 
    2 & 52.8\%
    (218) & 48.9\%
    (305) & 48.9\%
    (315) & 46.6\%
    (262) & 48.7\%
    (189) & 42.1\%
    (133) \\ 
    3 & 54.5\%
    (123) & 46.0\%
    (200) & 49.6\%
    (250) & 45.6\%
    (248) & 44.4\%
    (214) & 44.8\%
    (174) \\ 
    4 & 56.9\%
    (65) & 54.5\%
    (145) & 47.2\%
    (193) & 44.7\%
    (228) & 55.2\%
    (221) & 50.5\%
    (208) \\ 
    5 & 47.4\%
    (38) & 49.4\%
    (83) & 47.1\%
    (136) & 50.9\%
    (175) & 45.2\%
    (177) & 46.0\%
    (202) \\ 
  \bottomrule
\end{longtable}

\hypertarget{literature-review}{%
\section{Literature review}\label{literature-review}}

There have been only a handful of studies of the distribution of games
played in a series, most of which assume a constant probability \(p\) of
a given team winning a game in the series, regardless of the series
state. (Mosteller 1952) observed that the American League had dominated
the National League in Major League Baseball's (MLB) World Series
matchups, implying that matchups should not modeled with \(p = 0.5\).
Mosteller proposed three approaches for identifying the optimal constant
probability value of the stronger team in the World Series, finding
\(p \approx 0.65\). in each case: (1) a ``method of moments'' approach
in which one solves for \(p\). from the empirical average number of
games won by the loser of the series; (2) maximizing the likelihood that
the sample would have been drawn from a population in which the
probability of a team winning a game is constant across the series, and
(3) minimizing the chi-square goodness of fit estimate for \(p\).

(Chance 2020) re-examines the constant probability notion in Major
League Baseball's World Series (1923--2018), the National Basketball
Association's Finals (1951--2018), and the National Hockey League's
Stanley Cup (1939--2018). Chance applies finds strong evidence against
the null hypothesis of \(p = 0.5\) in the MLB and NHL championship
series when applying Mosteller's first and second methods.\footnote{Chance
  goes on to outline a conditional probability framework (likelihood of
  winning a game given the series state) which can exactly explain the
  distribution of the number of games played.} Chance's work is closely
related to mys and, in fact, provides a guide for the first part of my
investigation.

Momentum, one of most discussed topics in sports analytics, goes
hand-in-hand with a discussion of the nature of series
outcomes.\footnote{We often use use ``streaks'' and momentum
  interchangeably, but as (Steeger, Dulin, and Gonzalez 2021) note,
  momentum implies dependence between events, whereas streaking does
  not.} Two opposing fallacies are observed in the context of momentum:
the ``gambler's fallacy'' (negative recency) and ``hot hand fallacy''
(positive recency). Per (Ayton and Fischer 2004), negative recency is
``the belief that, for random events, runs of a particular outcome
\ldots{} will be balanced by a tendency for the opposite outcome'',
while positive recency is the expectation of observing future results
that match recent results.

Studying both player streaks and team streaks in basketball, in both
observational and controlled settings. (Gilovich, Vallone, and Tversky
1985) do not find evidence for the hot hand phenomenon. Recently ,
(Miller and Sanjurjo 2018) refute (Gilovich, Vallone, and Tversky
1985)'s conclusions, finding mathematical evidence that seems to support
negative recency. Specifically, they find that a ``bias exists in a
common measure of the conditional dependence of present outcomes on
streaks of past outcomes in sequential data'' (streak selection bias)
that imply that, under i.i.d. conditions, ``the proportion of successes
among the outcomes that immediately follow a streak of consecutive
successes is expected to be strictly less than the underlying
(conditional) probability of success''. I agree with Miller's findings,
accounting for the streak selection bias in my study of momentum.

Despite the plethora of existing research on games played in a series
and momentum in sports, these topics have yet to be investigated heavily
in esports. Work has been done to examine in-round win probability in
other FPS titles such as Counter-Strike ((Xenopoulos, Freeman, and Silva
2022)) and Valorant ((DeRover 2021)), both of which are round-based like
CoD SnD. However, research on round-level trends is sparse, perhaps
because games like Counter-Strike and Valorant both have economic
aspects that can create clear advantages on side in a given round, given
how prior rounds played out.\footnote{Additionally, both Counter-Strike
  and Valorant have overtime rules and blocked offensive/defensive roles
  (i.e.~playing either offense or defense for many consecutive rounds).}

\hypertarget{methodology-results-and-discussion}{%
\section{Methodology, results, and
discussion}\label{methodology-results-and-discussion}}

First, I investigate the constant probability assumption and the
distribution of rounds played in a series. Afterwards, I investigate
momentum, building on my learnings from the constant probability
assumption analysis.

\hypertarget{distribution-of-rounds-played}{%
\subsection{Distribution of rounds
played}\label{distribution-of-rounds-played}}

The general formula for the probability \(P_E(i)\) that a best-of-\(s\)
series lasts \(i\) rounds given constant probability \(p\) of one
team\footnote{If \(p > 0.5\), then I might say that this team is the
  better team (known omnisciently).} winning each round is

\[
q = 1 - p, s_1 = \frac{s - 1}{2}, s_2 = \frac{s + 1}{2}, s_2^{'} = 1 - s_2
\]

\begin{equation}\protect\hypertarget{eq-series-length}{}{
P_E(i) = \frac{(i - 1)!}{s_1!(i - s_2)!}(p^{s_2}q^{s_2^{'}} + p^{s_2^{'}}q^{s_2}).
}\label{eq:series-length}\end{equation}

Table \ref{tbl:cod-prob-series-lasting-i-rounds} shows the expected
proportion for \(s = 11\) and the naive assumption that \(p = 0.5\),
along with the observed frequencies observed in CoD SnD, as both a count
\(N_O(i)\) (of series lasting \(i\) rounds) and a proportion \(P_O(i)\)
of all series.

\begin{longtable}{rrrr}
  \caption{The probabilities that a best-of-11 series lasts $i$ rounds ($Pr(i)$, where $i = {6, 7, 8, 9, 10, 11}$) under the assumption that each team has a 50% probability ($p = 0.5$) of winning each round.}\label{tbl:cod-prob-series-lasting-i-rounds} \\
  \toprule
  &  & \multicolumn{2}{c}{Observed} \\ 
  \cmidrule(lr){3-4}
  Series lasts $i$ rounds & $P_E(i)$ & $P_O(i)$ & $N_O(i)$ \\ 
  \midrule
  6 & $3.1\%$ & $4.7\%$ & 40 \\ 
  7 & $9.4\%$ & $11.9\%$ & 101 \\ 
  8 & $16.4\%$ & $16.5\%$ & 141 \\ 
  9 & $21.9\%$ & $21.7\%$ & 185 \\ 
  10 & $24.6\%$ & $21.5\%$ & 183 \\ 
  11 & $24.6\%$ & $23.7\%$ & 202 \\ 
  \bottomrule
\end{longtable}

The probabilities that a best-of-11 series lasts \(i\) rounds (\(P(i)\),
where \(i \in R = [6, 7, 8, 9, 10, 11]\)) under the assumption that each
team has a 50\% probability (\(p = 0.5\)) of winning each game
\{\#tbl-prob-series-lasting-i-rounds\} along with the observed
frequencies of series lasting \(i\) rounds, expressed as a percentage
\(P_O(i)\) and as a count (\(N_O\))

Calculating the chi-square goodness of fit statistic

\begin{equation}\protect\hypertarget{eq-series-length}{}{
\chi^2 = \sum^{11}_{i=6} \frac{(P_O(i) - P_E(i))^2}{P_E(i)}, i \in R = [6, 7, 8, 9, 10, 11].
}\label{eq:chi-squ-test}\end{equation}

as 16.0 (p-value of 0.0068), I can comfortably reject the constant
probability null hypothesis, event at a confidence level of
\(\alpha = 0.01\).

Table \ref{tbl:mosteller-methods-results} shows the alternate values for
the constant round win probability \(p\) for winning a given round in a
CoD SnD that I find when applying the three methods suggested by
Mosteller (1952). Each is approximately or equal to 0.575, and, when
applying Equation \ref{eq:chi-squ-test}, each results in a \(\chi^2\)
value for which I cannot reject the constant probability null
hypothesis.

\begin{longtable}[]{@{}lrr@{}}
  \caption{Alternate estimates of the constant probability \(p\) for winning a given round in a CoD SnD, applying the three methods suggested by Mosteller (1952), in addition to the naive \(p = 0.5\).}\label{tbl:mosteller-methods-results} \\
  \toprule()
    Method & \(p\) & \(\chi^2\) (p-value) \\
    \midrule()
    \endfirsthead
    \toprule()
    Method & \(p\) & \(\chi^2\) (p-value) \\
    \midrule()
  \endhead
  0. Naive & 0.5000 & 16.0 (\textless=0.01) \\
  1. Method of moments & 0.5725 & 3.6 (0.6) \\
  2. Maximum likelihood & 0.5750 & 3.5 (0.62) \\
  3. Minimum \(\chi^2\) & 0.5775 & 3.5 (0.62) \\
  \bottomrule()
\end{longtable}

Table \ref{tbl:expected-series-lengths-alternative-ps} shows the new
\(P_E(i)\) when re-applying Equation \ref{eq:series-length} for each new
\(p\). I observe that \(P_E(i)\) is notably smaller for \(p_{1,2,3}\)
when \(i \in [10, 11]\) and higher for \(i \in [6, 7]\), more closely
matching \(P_O(i)\).

\begin{longtable}[]{@{}
  >{\raggedleft\arraybackslash}p{(\columnwidth - 10\tabcolsep) * \real{0.2427}}
  >{\raggedleft\arraybackslash}p{(\columnwidth - 10\tabcolsep) * \real{0.1456}}
  >{\raggedleft\arraybackslash}p{(\columnwidth - 10\tabcolsep) * \real{0.1748}}
  >{\raggedleft\arraybackslash}p{(\columnwidth - 10\tabcolsep) * \real{0.1650}}
  >{\raggedleft\arraybackslash}p{(\columnwidth - 10\tabcolsep) * \real{0.1748}}
  >{\raggedleft\arraybackslash}p{(\columnwidth - 10\tabcolsep) * \real{0.0971}}@{}}
\caption{The probabilities that a best-of-11 series lasts $i$ rounds under various assumptions for $p$.}\label{tbl:expected-series-lengths-alternative-ps} \\
\toprule()
\begin{minipage}[b]{\linewidth}\raggedleft
Series lasts \(i\) rounds
\end{minipage} & \begin{minipage}[b]{\linewidth}\raggedleft
0. \(p\) = 0.5
\end{minipage} & \begin{minipage}[b]{\linewidth}\raggedleft
1. \(p\) = 0.5725
\end{minipage} & \begin{minipage}[b]{\linewidth}\raggedleft
2. \(p\) = 0.575
\end{minipage} & \begin{minipage}[b]{\linewidth}\raggedleft
3. \(p\) = 0.5775
\end{minipage} & \begin{minipage}[b]{\linewidth}\raggedleft
\(P_O(i)\)
\end{minipage} \\
\midrule()
\endfirsthead
\toprule()
\begin{minipage}[b]{\linewidth}\raggedleft
Series lasts \(i\) rounds
\end{minipage} & \begin{minipage}[b]{\linewidth}\raggedleft
0. \(p\) = 0.5
\end{minipage} & \begin{minipage}[b]{\linewidth}\raggedleft
1. \(p\) = 0.5725
\end{minipage} & \begin{minipage}[b]{\linewidth}\raggedleft
2. \(p\) = 0.575
\end{minipage} & \begin{minipage}[b]{\linewidth}\raggedleft
3. \(p\) = 0.5775
\end{minipage} & \begin{minipage}[b]{\linewidth}\raggedleft
\(P_O(i)\)
\end{minipage} \\
\midrule()
\endhead
6 & 3.1\% & 4.1\% & 4.2\% & 4.3\% & 4.7\% \\
7 & 9.4\% & 11.1\% & 11.2\% & 11.4\% & 11.9\% \\
8 & 16.4\% & 17.7\% & 17.8\% & 17.9\% & 16.5\% \\
9 & 21.9\% & 21.8\% & 21.8\% & 21.8\% & 21.7\% \\
10 & 24.6\% & 23.1\% & 23.0\% & 22.9\% & 21.5\% \\
11 & 24.6\% & 22.1\% & 22.0\% & 21.8\% & 23.7\% \\
\bottomrule()
\end{longtable}

So I conclude that the constant round win probability assumption can be
valid in CoD SnD series with the appropriate choice of \(p\)
(\(\approx 0.575\)).

\hypertarget{references}{%
\section*{References}\label{references}}
\addcontentsline{toc}{section}{References}

\hypertarget{refs}{}
\begin{CSLReferences}{1}{0}
\leavevmode\vadjust pre{\hypertarget{ref-ayton2004}{}}%
Ayton, Peter, and Ilan Fischer. 2004. {``The Hot Hand Fallacy and the
Gambler{'}s Fallacy: Two Faces of Subjective Randomness?''} \emph{Memory
\& Cognition} 32 (8): 13691378.

\leavevmode\vadjust pre{\hypertarget{ref-chance2020}{}}%
Chance, Don. 2020. {``Conditional Probability and the Length of a
Championship Series in Baseball, Basketball, and Hockey.''}
\emph{Journal of Sports Analytics} 6 (2): 111--27.
\url{https://doi.org/10.3233/JSA-200422}.

\leavevmode\vadjust pre{\hypertarget{ref-derover2021}{}}%
DeRover, DeMars. 2021. {``Round Win Probabilities Based on Who's Alive +
Time.''}
\url{https://www.reddit.com/r/VALORANT/comments/n3lpoo/round_win_probabilities_based_on_whos_alive_time/}.

\leavevmode\vadjust pre{\hypertarget{ref-gilovich1985}{}}%
Gilovich, Thomas, Robert Vallone, and Amos Tversky. 1985. {``The Hot
Hand in Basketball: On the Misperception of Random Sequences.''}
\emph{Cognitive Psychology} 17 (3): 295314.

\leavevmode\vadjust pre{\hypertarget{ref-miller2018}{}}%
Miller, Joshua B., and Adam Sanjurjo. 2018. {``Surprised by the Hot Hand
Fallacy? A Truth in the Law of Small Numbers.''} \emph{Econometrica} 86
(6): 20192047.

\leavevmode\vadjust pre{\hypertarget{ref-mosteller1952}{}}%
Mosteller, Frederick. 1952. {``The World Series Competition.''}
\emph{Journal of the American Statistical Association} 47 (259): 355380.

\leavevmode\vadjust pre{\hypertarget{ref-steeger2021}{}}%
Steeger, Gregory M., Johnathon L. Dulin, and Gerardo O. Gonzalez. 2021.
{``Winning and Losing Streaks in the National Hockey League: Are Teams
Experiencing Momentum or Are Games a Sequence of Random Events?''}
\emph{Journal of Quantitative Analysis in Sports} 17 (3): 155170.

\leavevmode\vadjust pre{\hypertarget{ref-xenopoulos2022}{}}%
Xenopoulos, Peter, William Robert Freeman, and Claudio Silva. 2022.
{``Analyzing the Differences Between Professional and Amateur Esports
Through Win Probability.''} In, 34183427.

\end{CSLReferences}

\bibliographystyle{unsrt}
\bibliography{references.bib}


\end{document}
